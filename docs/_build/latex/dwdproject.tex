%% Generated by Sphinx.
\def\sphinxdocclass{report}
\documentclass[letterpaper,10pt,english]{sphinxmanual}
\ifdefined\pdfpxdimen
   \let\sphinxpxdimen\pdfpxdimen\else\newdimen\sphinxpxdimen
\fi \sphinxpxdimen=.75bp\relax
\ifdefined\pdfimageresolution
    \pdfimageresolution= \numexpr \dimexpr1in\relax/\sphinxpxdimen\relax
\fi
%% let collapsible pdf bookmarks panel have high depth per default
\PassOptionsToPackage{bookmarksdepth=5}{hyperref}

\PassOptionsToPackage{warn}{textcomp}
\usepackage[utf8]{inputenc}
\ifdefined\DeclareUnicodeCharacter
% support both utf8 and utf8x syntaxes
  \ifdefined\DeclareUnicodeCharacterAsOptional
    \def\sphinxDUC#1{\DeclareUnicodeCharacter{"#1}}
  \else
    \let\sphinxDUC\DeclareUnicodeCharacter
  \fi
  \sphinxDUC{00A0}{\nobreakspace}
  \sphinxDUC{2500}{\sphinxunichar{2500}}
  \sphinxDUC{2502}{\sphinxunichar{2502}}
  \sphinxDUC{2514}{\sphinxunichar{2514}}
  \sphinxDUC{251C}{\sphinxunichar{251C}}
  \sphinxDUC{2572}{\textbackslash}
\fi
\usepackage{cmap}
\usepackage[T1]{fontenc}
\usepackage{amsmath,amssymb,amstext}
\usepackage{babel}



\usepackage{tgtermes}
\usepackage{tgheros}
\renewcommand{\ttdefault}{txtt}



\usepackage[Bjarne]{fncychap}
\usepackage{sphinx}

\fvset{fontsize=auto}
\usepackage{geometry}


% Include hyperref last.
\usepackage{hyperref}
% Fix anchor placement for figures with captions.
\usepackage{hypcap}% it must be loaded after hyperref.
% Set up styles of URL: it should be placed after hyperref.
\urlstyle{same}

\addto\captionsenglish{\renewcommand{\contentsname}{Contents:}}

\usepackage{sphinxmessages}




\title{DWD Project}
\date{Jun 05, 2022}
\release{}
\author{Vitali Krilov}
\newcommand{\sphinxlogo}{\vbox{}}
\renewcommand{\releasename}{}
\makeindex
\begin{document}

\pagestyle{empty}
\sphinxmaketitle
\pagestyle{plain}
\sphinxtableofcontents
\pagestyle{normal}
\phantomsection\label{\detokenize{index::doc}}



\chapter{DWD\_Project}
\label{\detokenize{modules:dwd-project}}\label{\detokenize{modules::doc}}

\section{DwdDataPrep module}
\label{\detokenize{DwdDataPrep:module-DwdDataPrep}}\label{\detokenize{DwdDataPrep:dwddataprep-module}}\label{\detokenize{DwdDataPrep::doc}}\index{module@\spxentry{module}!DwdDataPrep@\spxentry{DwdDataPrep}}\index{DwdDataPrep@\spxentry{DwdDataPrep}!module@\spxentry{module}}\index{Reader (class in DwdDataPrep)@\spxentry{Reader}\spxextra{class in DwdDataPrep}}

\begin{fulllineitems}
\phantomsection\label{\detokenize{DwdDataPrep:DwdDataPrep.Reader}}\pysiglinewithargsret{\sphinxbfcode{\sphinxupquote{class }}\sphinxcode{\sphinxupquote{DwdDataPrep.}}\sphinxbfcode{\sphinxupquote{Reader}}}{\emph{\DUrole{n}{path\_to\_txt}}, \emph{\DUrole{n}{local\_path}}, \emph{\DUrole{n}{data\_type}}}{}
\sphinxAtStartPar
Bases: \sphinxcode{\sphinxupquote{object}}
\begin{quote}\begin{description}
\item[{Description}] \leavevmode
\sphinxAtStartPar
This class will read all the descriptions / informations of all the stations and will generate needed data.

\end{description}\end{quote}
\index{\_\_init\_\_() (DwdDataPrep.Reader method)@\spxentry{\_\_init\_\_()}\spxextra{DwdDataPrep.Reader method}}

\begin{fulllineitems}
\phantomsection\label{\detokenize{DwdDataPrep:DwdDataPrep.Reader.__init__}}\pysiglinewithargsret{\sphinxbfcode{\sphinxupquote{\_\_init\_\_}}}{\emph{\DUrole{n}{path\_to\_txt}}, \emph{\DUrole{n}{local\_path}}, \emph{\DUrole{n}{data\_type}}}{}~\begin{quote}\begin{description}
\item[{Parameters}] \leavevmode\begin{itemize}
\item {} 
\sphinxAtStartPar
\sphinxstyleliteralstrong{\sphinxupquote{path\_to\_txt}} \textendash{} \sphinxstylestrong{as string}: Is the path to the txt of station descriptions / informations

\item {} 
\sphinxAtStartPar
\sphinxstyleliteralstrong{\sphinxupquote{local\_path}} \textendash{} \sphinxstylestrong{as string}: Is the path to the extracted\_files

\item {} 
\sphinxAtStartPar
\sphinxstyleliteralstrong{\sphinxupquote{data\_type}} \textendash{} \sphinxstylestrong{as string}: Is the type of data (Example: “TU” or “SOLAR”)

\end{itemize}

\end{description}\end{quote}

\end{fulllineitems}

\index{data\_prep\_for\_new\_location() (DwdDataPrep.Reader method)@\spxentry{data\_prep\_for\_new\_location()}\spxextra{DwdDataPrep.Reader method}}

\begin{fulllineitems}
\phantomsection\label{\detokenize{DwdDataPrep:DwdDataPrep.Reader.data_prep_for_new_location}}\pysiglinewithargsret{\sphinxbfcode{\sphinxupquote{data\_prep\_for\_new\_location}}}{\emph{\DUrole{n}{start\_date}}, \emph{\DUrole{n}{end\_date}}, \emph{\DUrole{n}{x\_coordinate}}, \emph{\DUrole{n}{y\_coordinate}}, \emph{\DUrole{n}{z\_coordinate}}}{}~\begin{quote}\begin{description}
\item[{Description}] \leavevmode
\sphinxAtStartPar
Will prepare all the data that is available in timedelta for a new location

\item[{Param}] \leavevmode
\sphinxAtStartPar
x\_coordinate: \sphinxstylestrong{as float}: (GeoLaenge) for the new location

\item[{Param}] \leavevmode
\sphinxAtStartPar
y\_coordinate: \sphinxstylestrong{as float}: (GeoBreite) for the new location

\item[{Param}] \leavevmode
\sphinxAtStartPar
z\_coordinate: \sphinxstylestrong{as float}: 0 for now (maybe needed in future) for the new location

\item[{Returns}] \leavevmode
\sphinxAtStartPar
x\_active\_in\_date, y\_active\_in\_date, z\_active\_in\_date, activ\_id\_in\_date, station\_list: \sphinxstylestrong{as list \sphinxhyphen{} station\_list as dict}

\end{description}\end{quote}

\end{fulllineitems}

\index{get\_active\_stations() (DwdDataPrep.Reader method)@\spxentry{get\_active\_stations()}\spxextra{DwdDataPrep.Reader method}}

\begin{fulllineitems}
\phantomsection\label{\detokenize{DwdDataPrep:DwdDataPrep.Reader.get_active_stations}}\pysiglinewithargsret{\sphinxbfcode{\sphinxupquote{get\_active\_stations}}}{}{}~\begin{quote}\begin{description}
\item[{Description}] \leavevmode
\sphinxAtStartPar
Will create some important arrays (all active stations, not active stations and stations with no data) for past 10 days

\item[{Returns}] \leavevmode
\sphinxAtStartPar
x\_active, y\_active, z\_active, active\_id, x\_not\_active, y\_not\_active, z\_not\_active, not\_activ\_id, x\_no\_data, y\_no\_data, z\_no\_data, no\_data\_id: \sphinxstylestrong{all as array}

\end{description}\end{quote}

\end{fulllineitems}

\index{get\_active\_stations\_in\_date() (DwdDataPrep.Reader method)@\spxentry{get\_active\_stations\_in\_date()}\spxextra{DwdDataPrep.Reader method}}

\begin{fulllineitems}
\phantomsection\label{\detokenize{DwdDataPrep:DwdDataPrep.Reader.get_active_stations_in_date}}\pysiglinewithargsret{\sphinxbfcode{\sphinxupquote{get\_active\_stations\_in\_date}}}{\emph{\DUrole{n}{start\_date}}, \emph{\DUrole{n}{end\_date}}}{}~\begin{quote}\begin{description}
\item[{Description}] \leavevmode
\sphinxAtStartPar
Will create some important arrays (all active stations, not active stations and stations with no data) in timedelta

\item[{Param}] \leavevmode
\sphinxAtStartPar
start\_date: \sphinxstylestrong{as int}: YYYYMMDDHHMM

\item[{Param}] \leavevmode
\sphinxAtStartPar
end\_date: \sphinxstylestrong{as int}: YYYYMMDDHHMM

\item[{Returns}] \leavevmode
\sphinxAtStartPar
x\_active, y\_active, z\_active, active\_id: \sphinxstylestrong{all as array}

\end{description}\end{quote}

\end{fulllineitems}

\index{get\_station\_ids() (DwdDataPrep.Reader method)@\spxentry{get\_station\_ids()}\spxextra{DwdDataPrep.Reader method}}

\begin{fulllineitems}
\phantomsection\label{\detokenize{DwdDataPrep:DwdDataPrep.Reader.get_station_ids}}\pysiglinewithargsret{\sphinxbfcode{\sphinxupquote{get\_station\_ids}}}{}{}~\begin{quote}\begin{description}
\item[{Description}] \leavevmode
\sphinxAtStartPar
Will create an array with all the station\_ids with a prefix

\item[{Returns}] \leavevmode
\sphinxAtStartPar
station\_ids\_array: \sphinxstyleemphasis{as array*}

\end{description}\end{quote}

\end{fulllineitems}

\index{get\_station\_ids\_in\_date() (DwdDataPrep.Reader method)@\spxentry{get\_station\_ids\_in\_date()}\spxextra{DwdDataPrep.Reader method}}

\begin{fulllineitems}
\phantomsection\label{\detokenize{DwdDataPrep:DwdDataPrep.Reader.get_station_ids_in_date}}\pysiglinewithargsret{\sphinxbfcode{\sphinxupquote{get\_station\_ids\_in\_date}}}{\emph{\DUrole{n}{start\_date}}, \emph{\DUrole{n}{end\_date}}}{}~\begin{quote}\begin{description}
\item[{Description}] \leavevmode
\sphinxAtStartPar
Will create a mask for active stations in timedelta

\item[{Param}] \leavevmode
\sphinxAtStartPar
start\_date: \sphinxstylestrong{as int}: YYYYMMDDHHMM

\item[{Param}] \leavevmode
\sphinxAtStartPar
end\_date: \sphinxstylestrong{as int}: YYYYMMDDHHMM

\item[{Returns}] \leavevmode
\sphinxAtStartPar
activ\_stations\_in\_date\_array: \sphinxstyleemphasis{as array*} (is a mask)

\end{description}\end{quote}

\end{fulllineitems}

\index{get\_station\_information() (DwdDataPrep.Reader method)@\spxentry{get\_station\_information()}\spxextra{DwdDataPrep.Reader method}}

\begin{fulllineitems}
\phantomsection\label{\detokenize{DwdDataPrep:DwdDataPrep.Reader.get_station_information}}\pysiglinewithargsret{\sphinxbfcode{\sphinxupquote{get\_station\_information}}}{}{}~\begin{quote}\begin{description}
\item[{Description}] \leavevmode
\sphinxAtStartPar
Will create an array with all the information from the informations\sphinxhyphen{}txt

\item[{Returns}] \leavevmode
\sphinxAtStartPar
station\_information: \sphinxstylestrong{as array (“i4”, “i4”, “i4”, “i4”, “f4”, “f4”, “S20”, “S20”)}

\end{description}\end{quote}

\end{fulllineitems}

\index{read\_station\_list() (DwdDataPrep.Reader method)@\spxentry{read\_station\_list()}\spxextra{DwdDataPrep.Reader method}}

\begin{fulllineitems}
\phantomsection\label{\detokenize{DwdDataPrep:DwdDataPrep.Reader.read_station_list}}\pysiglinewithargsret{\sphinxbfcode{\sphinxupquote{read\_station\_list}}}{}{}~\begin{quote}\begin{description}
\item[{Description}] \leavevmode
\sphinxAtStartPar
Will create all the station objects in a dictionary

\item[{Returns}] \leavevmode
\sphinxAtStartPar
stations: \sphinxstylestrong{as dict}

\end{description}\end{quote}

\end{fulllineitems}


\end{fulllineitems}

\index{Station (class in DwdDataPrep)@\spxentry{Station}\spxextra{class in DwdDataPrep}}

\begin{fulllineitems}
\phantomsection\label{\detokenize{DwdDataPrep:DwdDataPrep.Station}}\pysiglinewithargsret{\sphinxbfcode{\sphinxupquote{class }}\sphinxcode{\sphinxupquote{DwdDataPrep.}}\sphinxbfcode{\sphinxupquote{Station}}}{\emph{\DUrole{n}{stations\_array}}, \emph{\DUrole{n}{local\_path}}, \emph{\DUrole{n}{data\_type}}}{}
\sphinxAtStartPar
Bases: \sphinxcode{\sphinxupquote{object}}
\begin{quote}\begin{description}
\item[{Description}] \leavevmode
\sphinxAtStartPar
This class will generate stations as objects with all the given informations.

\end{description}\end{quote}
\index{\_\_init\_\_() (DwdDataPrep.Station method)@\spxentry{\_\_init\_\_()}\spxextra{DwdDataPrep.Station method}}

\begin{fulllineitems}
\phantomsection\label{\detokenize{DwdDataPrep:DwdDataPrep.Station.__init__}}\pysiglinewithargsret{\sphinxbfcode{\sphinxupquote{\_\_init\_\_}}}{\emph{\DUrole{n}{stations\_array}}, \emph{\DUrole{n}{local\_path}}, \emph{\DUrole{n}{data\_type}}}{}~\begin{quote}\begin{description}
\item[{Parameters}] \leavevmode\begin{itemize}
\item {} 
\sphinxAtStartPar
\sphinxstyleliteralstrong{\sphinxupquote{stations\_array}} \textendash{} \sphinxstylestrong{as array}. Is generated by Reader(…).read\_station\_list()

\item {} 
\sphinxAtStartPar
\sphinxstyleliteralstrong{\sphinxupquote{local\_path}} \textendash{} \sphinxstylestrong{as string}. Is getting from Reader class

\item {} 
\sphinxAtStartPar
\sphinxstyleliteralstrong{\sphinxupquote{data\_type}} \textendash{} \sphinxstylestrong{as string}. Ss getting from Reader class

\end{itemize}

\end{description}\end{quote}

\end{fulllineitems}

\index{check\_activ\_in\_date() (DwdDataPrep.Station method)@\spxentry{check\_activ\_in\_date()}\spxextra{DwdDataPrep.Station method}}

\begin{fulllineitems}
\phantomsection\label{\detokenize{DwdDataPrep:DwdDataPrep.Station.check_activ_in_date}}\pysiglinewithargsret{\sphinxbfcode{\sphinxupquote{check\_activ\_in\_date}}}{\emph{\DUrole{n}{start\_date}}, \emph{\DUrole{n}{end\_date}}}{}
\sphinxAtStartPar
\sphinxstylestrong{Description:} Will check the station activity in timedelta (True: is activ in timedelta, False: not activ in timedelta)
\begin{quote}\begin{description}
\item[{Parameters}] \leavevmode\begin{itemize}
\item {} 
\sphinxAtStartPar
\sphinxstyleliteralstrong{\sphinxupquote{start\_date}} \textendash{} \sphinxstylestrong{as int}: YYYYMMDDHHMM

\item {} 
\sphinxAtStartPar
\sphinxstyleliteralstrong{\sphinxupquote{end\_date}} \textendash{} \sphinxstylestrong{as int}: YYYYMMDDHHMM

\end{itemize}

\item[{Returns}] \leavevmode
\sphinxAtStartPar
Boolean

\end{description}\end{quote}

\end{fulllineitems}

\index{check\_activitiy() (DwdDataPrep.Station method)@\spxentry{check\_activitiy()}\spxextra{DwdDataPrep.Station method}}

\begin{fulllineitems}
\phantomsection\label{\detokenize{DwdDataPrep:DwdDataPrep.Station.check_activitiy}}\pysiglinewithargsret{\sphinxbfcode{\sphinxupquote{check\_activitiy}}}{}{}
\sphinxAtStartPar
\sphinxstylestrong{Description:} Will check the activity of the station (was this station activ past 10 days?)
\begin{quote}\begin{description}
\item[{Returns}] \leavevmode
\sphinxAtStartPar
\sphinxstylestrong{Boolean}

\end{description}\end{quote}

\end{fulllineitems}

\index{check\_data() (DwdDataPrep.Station method)@\spxentry{check\_data()}\spxextra{DwdDataPrep.Station method}}

\begin{fulllineitems}
\phantomsection\label{\detokenize{DwdDataPrep:DwdDataPrep.Station.check_data}}\pysiglinewithargsret{\sphinxbfcode{\sphinxupquote{check\_data}}}{}{}
\sphinxAtStartPar
\sphinxstylestrong{Description:} Is checking if this station got any data available (0 == no data, 1 == data available)
\begin{quote}\begin{description}
\item[{Returns}] \leavevmode
\sphinxAtStartPar
0 or 1: \sphinxstylestrong{as int}

\end{description}\end{quote}

\end{fulllineitems}

\index{format\_number\_to\_date() (DwdDataPrep.Station static method)@\spxentry{format\_number\_to\_date()}\spxextra{DwdDataPrep.Station static method}}

\begin{fulllineitems}
\phantomsection\label{\detokenize{DwdDataPrep:DwdDataPrep.Station.format_number_to_date}}\pysiglinewithargsret{\sphinxbfcode{\sphinxupquote{static }}\sphinxbfcode{\sphinxupquote{format\_number\_to\_date}}}{\emph{\DUrole{n}{number}}}{}
\sphinxAtStartPar
\sphinxstylestrong{Description:} formating the date yyymmddhhmm \textendash{}\textgreater{} YYYY\sphinxhyphen{}MM\sphinxhyphen{}DD
\begin{quote}\begin{description}
\item[{Parameters}] \leavevmode
\sphinxAtStartPar
\sphinxstyleliteralstrong{\sphinxupquote{number}} \textendash{} \sphinxstylestrong{as int}.

\item[{Returns}] \leavevmode
\sphinxAtStartPar
date: \sphinxstylestrong{as string}.

\end{description}\end{quote}

\end{fulllineitems}

\index{generate\_tu\_data\_path() (DwdDataPrep.Station method)@\spxentry{generate\_tu\_data\_path()}\spxextra{DwdDataPrep.Station method}}

\begin{fulllineitems}
\phantomsection\label{\detokenize{DwdDataPrep:DwdDataPrep.Station.generate_tu_data_path}}\pysiglinewithargsret{\sphinxbfcode{\sphinxupquote{generate\_tu\_data\_path}}}{}{}
\sphinxAtStartPar
\sphinxstylestrong{Description:} Will generate all the local paths for this station
\begin{quote}\begin{description}
\item[{Returns}] \leavevmode
\sphinxAtStartPar
local\_list: \sphinxstylestrong{as list}

\end{description}\end{quote}

\end{fulllineitems}

\index{generate\_tu\_data\_path\_date() (DwdDataPrep.Station method)@\spxentry{generate\_tu\_data\_path\_date()}\spxextra{DwdDataPrep.Station method}}

\begin{fulllineitems}
\phantomsection\label{\detokenize{DwdDataPrep:DwdDataPrep.Station.generate_tu_data_path_date}}\pysiglinewithargsret{\sphinxbfcode{\sphinxupquote{generate\_tu\_data\_path\_date}}}{\emph{\DUrole{n}{start\_date}}, \emph{\DUrole{n}{end\_date}}}{}
\sphinxAtStartPar
\sphinxstylestrong{Description:} Will generate all the data paths in timedelta. If no data available for this station, returns False
\begin{quote}\begin{description}
\item[{Parameters}] \leavevmode\begin{itemize}
\item {} 
\sphinxAtStartPar
\sphinxstyleliteralstrong{\sphinxupquote{start\_date}} \textendash{} \sphinxstylestrong{as int}: YYYYMMDDHHMM

\item {} 
\sphinxAtStartPar
\sphinxstyleliteralstrong{\sphinxupquote{end\_date}} \textendash{} \sphinxstylestrong{as int}: YYYYMMDDHHMM

\end{itemize}

\item[{Returns}] \leavevmode
\sphinxAtStartPar
generate\_tu\_data\_path\_date: \sphinxstylestrong{as list} or False

\end{description}\end{quote}

\end{fulllineitems}

\index{generate\_tu\_date\_array() (DwdDataPrep.Station method)@\spxentry{generate\_tu\_date\_array()}\spxextra{DwdDataPrep.Station method}}

\begin{fulllineitems}
\phantomsection\label{\detokenize{DwdDataPrep:DwdDataPrep.Station.generate_tu_date_array}}\pysiglinewithargsret{\sphinxbfcode{\sphinxupquote{generate\_tu\_date\_array}}}{}{}
\sphinxAtStartPar
\sphinxstylestrong{Description:} Will generate all the data paths and all the timedelta the station was activ in
\begin{quote}\begin{description}
\item[{Returns}] \leavevmode
\sphinxAtStartPar
from\_list: \sphinxstylestrong{as list}, tu\_data\_path: \sphinxstylestrong{as list}

\end{description}\end{quote}

\end{fulllineitems}

\index{get\_bis\_datum() (DwdDataPrep.Station method)@\spxentry{get\_bis\_datum()}\spxextra{DwdDataPrep.Station method}}

\begin{fulllineitems}
\phantomsection\label{\detokenize{DwdDataPrep:DwdDataPrep.Station.get_bis_datum}}\pysiglinewithargsret{\sphinxbfcode{\sphinxupquote{get\_bis\_datum}}}{}{}
\sphinxAtStartPar
\sphinxstylestrong{Description:} Here you can get the \_\_bis\_datum information
\begin{quote}\begin{description}
\item[{Returns}] \leavevmode
\sphinxAtStartPar
\_\_bis\_datum: \sphinxstylestrong{as string}

\end{description}\end{quote}

\end{fulllineitems}

\index{get\_bundesland() (DwdDataPrep.Station method)@\spxentry{get\_bundesland()}\spxextra{DwdDataPrep.Station method}}

\begin{fulllineitems}
\phantomsection\label{\detokenize{DwdDataPrep:DwdDataPrep.Station.get_bundesland}}\pysiglinewithargsret{\sphinxbfcode{\sphinxupquote{get\_bundesland}}}{}{}
\sphinxAtStartPar
\sphinxstylestrong{Description:} Here you can get the \_\_bundesland information
\begin{quote}\begin{description}
\item[{Returns}] \leavevmode
\sphinxAtStartPar
\_\_bundesland: \sphinxstylestrong{as string}

\end{description}\end{quote}

\end{fulllineitems}

\index{get\_geobreite() (DwdDataPrep.Station method)@\spxentry{get\_geobreite()}\spxextra{DwdDataPrep.Station method}}

\begin{fulllineitems}
\phantomsection\label{\detokenize{DwdDataPrep:DwdDataPrep.Station.get_geobreite}}\pysiglinewithargsret{\sphinxbfcode{\sphinxupquote{get\_geobreite}}}{}{}
\sphinxAtStartPar
\sphinxstylestrong{Description:} Here you can get the \_\_geoBreite information
\begin{quote}\begin{description}
\item[{Returns}] \leavevmode
\sphinxAtStartPar
\_\_geoBreite: \sphinxstylestrong{as int}

\end{description}\end{quote}

\end{fulllineitems}

\index{get\_geolaenge() (DwdDataPrep.Station method)@\spxentry{get\_geolaenge()}\spxextra{DwdDataPrep.Station method}}

\begin{fulllineitems}
\phantomsection\label{\detokenize{DwdDataPrep:DwdDataPrep.Station.get_geolaenge}}\pysiglinewithargsret{\sphinxbfcode{\sphinxupquote{get\_geolaenge}}}{}{}
\sphinxAtStartPar
\sphinxstylestrong{Description:} Here you can get the \_\_geoLaenge information
\begin{quote}\begin{description}
\item[{Returns}] \leavevmode
\sphinxAtStartPar
\_\_geoLaenge: \sphinxstylestrong{as int}

\end{description}\end{quote}

\end{fulllineitems}

\index{get\_local\_path() (DwdDataPrep.Station method)@\spxentry{get\_local\_path()}\spxextra{DwdDataPrep.Station method}}

\begin{fulllineitems}
\phantomsection\label{\detokenize{DwdDataPrep:DwdDataPrep.Station.get_local_path}}\pysiglinewithargsret{\sphinxbfcode{\sphinxupquote{get\_local\_path}}}{}{}
\sphinxAtStartPar
\sphinxstylestrong{Description:} Here you can get the local\_path information
\begin{quote}\begin{description}
\item[{Returns}] \leavevmode
\sphinxAtStartPar
\_\_local\_path: \sphinxstylestrong{as string}

\end{description}\end{quote}

\end{fulllineitems}

\index{get\_station\_id() (DwdDataPrep.Station method)@\spxentry{get\_station\_id()}\spxextra{DwdDataPrep.Station method}}

\begin{fulllineitems}
\phantomsection\label{\detokenize{DwdDataPrep:DwdDataPrep.Station.get_station_id}}\pysiglinewithargsret{\sphinxbfcode{\sphinxupquote{get\_station\_id}}}{}{}
\sphinxAtStartPar
\sphinxstylestrong{Description:} Here you can get the stations\_id information
\begin{quote}\begin{description}
\item[{Returns}] \leavevmode
\sphinxAtStartPar
\_\_stations\_id: \sphinxstylestrong{as string}

\end{description}\end{quote}

\end{fulllineitems}

\index{get\_station\_informations() (DwdDataPrep.Station method)@\spxentry{get\_station\_informations()}\spxextra{DwdDataPrep.Station method}}

\begin{fulllineitems}
\phantomsection\label{\detokenize{DwdDataPrep:DwdDataPrep.Station.get_station_informations}}\pysiglinewithargsret{\sphinxbfcode{\sphinxupquote{get\_station\_informations}}}{}{}
\sphinxAtStartPar
\sphinxstylestrong{Description:} Will return all the informations of the station as a dict
\begin{quote}\begin{description}
\item[{Returns}] \leavevmode
\sphinxAtStartPar
station\_information\_dict: \sphinxstylestrong{as dict}

\end{description}\end{quote}

\end{fulllineitems}

\index{get\_stationshoehe() (DwdDataPrep.Station method)@\spxentry{get\_stationshoehe()}\spxextra{DwdDataPrep.Station method}}

\begin{fulllineitems}
\phantomsection\label{\detokenize{DwdDataPrep:DwdDataPrep.Station.get_stationshoehe}}\pysiglinewithargsret{\sphinxbfcode{\sphinxupquote{get\_stationshoehe}}}{}{}
\sphinxAtStartPar
\sphinxstylestrong{Description:} Here you can get the \_\_stationshoehe information
\begin{quote}\begin{description}
\item[{Returns}] \leavevmode
\sphinxAtStartPar
\_\_stationshoehe: \sphinxstylestrong{as int}

\end{description}\end{quote}

\end{fulllineitems}

\index{get\_stationsname() (DwdDataPrep.Station method)@\spxentry{get\_stationsname()}\spxextra{DwdDataPrep.Station method}}

\begin{fulllineitems}
\phantomsection\label{\detokenize{DwdDataPrep:DwdDataPrep.Station.get_stationsname}}\pysiglinewithargsret{\sphinxbfcode{\sphinxupquote{get\_stationsname}}}{}{}
\sphinxAtStartPar
\sphinxstylestrong{Description:} Here you can get the \_\_stationsname information
\begin{quote}\begin{description}
\item[{Returns}] \leavevmode
\sphinxAtStartPar
\_\_stationsname: \sphinxstylestrong{as string}

\end{description}\end{quote}

\end{fulllineitems}

\index{get\_von\_datum() (DwdDataPrep.Station method)@\spxentry{get\_von\_datum()}\spxextra{DwdDataPrep.Station method}}

\begin{fulllineitems}
\phantomsection\label{\detokenize{DwdDataPrep:DwdDataPrep.Station.get_von_datum}}\pysiglinewithargsret{\sphinxbfcode{\sphinxupquote{get\_von\_datum}}}{}{}
\sphinxAtStartPar
\sphinxstylestrong{Description:} Here you can get the von\_dotum information
\begin{quote}\begin{description}
\item[{Returns}] \leavevmode
\sphinxAtStartPar
\_\_von\_datum: \sphinxstylestrong{as string}

\end{description}\end{quote}

\end{fulllineitems}


\end{fulllineitems}

\index{Writer (class in DwdDataPrep)@\spxentry{Writer}\spxextra{class in DwdDataPrep}}

\begin{fulllineitems}
\phantomsection\label{\detokenize{DwdDataPrep:DwdDataPrep.Writer}}\pysiglinewithargsret{\sphinxbfcode{\sphinxupquote{class }}\sphinxcode{\sphinxupquote{DwdDataPrep.}}\sphinxbfcode{\sphinxupquote{Writer}}}{\emph{\DUrole{n}{path\_to\_txt}}, \emph{\DUrole{n}{local\_path}}, \emph{\DUrole{n}{data\_type}}}{}
\sphinxAtStartPar
Bases: \sphinxcode{\sphinxupquote{object}}
\begin{quote}\begin{description}
\item[{Description}] \leavevmode
\sphinxAtStartPar
This class write some json files. For faster loading times you should write this files after you downloaded and extracted your data.

\end{description}\end{quote}
\index{\_\_init\_\_() (DwdDataPrep.Writer method)@\spxentry{\_\_init\_\_()}\spxextra{DwdDataPrep.Writer method}}

\begin{fulllineitems}
\phantomsection\label{\detokenize{DwdDataPrep:DwdDataPrep.Writer.__init__}}\pysiglinewithargsret{\sphinxbfcode{\sphinxupquote{\_\_init\_\_}}}{\emph{\DUrole{n}{path\_to\_txt}}, \emph{\DUrole{n}{local\_path}}, \emph{\DUrole{n}{data\_type}}}{}~\begin{quote}\begin{description}
\item[{Parameters}] \leavevmode\begin{itemize}
\item {} 
\sphinxAtStartPar
\sphinxstyleliteralstrong{\sphinxupquote{path\_to\_txt}} \textendash{} \sphinxstylestrong{as string}: Is the path to the txt of station descriptions / informations

\item {} 
\sphinxAtStartPar
\sphinxstyleliteralstrong{\sphinxupquote{local\_path}} \textendash{} \sphinxstylestrong{as string}: Is the path to the extracted\_files

\item {} 
\sphinxAtStartPar
\sphinxstyleliteralstrong{\sphinxupquote{data\_type}} \textendash{} \sphinxstylestrong{as string}: Is the type of data (Example: “TU” or “SOLAR”)

\end{itemize}

\end{description}\end{quote}

\end{fulllineitems}

\index{write\_stations\_paths() (DwdDataPrep.Writer method)@\spxentry{write\_stations\_paths()}\spxextra{DwdDataPrep.Writer method}}

\begin{fulllineitems}
\phantomsection\label{\detokenize{DwdDataPrep:DwdDataPrep.Writer.write_stations_paths}}\pysiglinewithargsret{\sphinxbfcode{\sphinxupquote{write\_stations\_paths}}}{}{}~\begin{quote}\begin{description}
\item[{Description}] \leavevmode
\sphinxAtStartPar
Will write two json files in extracted\_files (dict\_tu\_data\_path and dict\_from\_list) with all the dates and all the paths for any station

\item[{Returns}] \leavevmode
\sphinxAtStartPar
path where the data was written

\end{description}\end{quote}

\end{fulllineitems}


\end{fulllineitems}



\section{DwdDataScrapper module}
\label{\detokenize{DwdDataScrapper:module-DwdDataScrapper}}\label{\detokenize{DwdDataScrapper:dwddatascrapper-module}}\label{\detokenize{DwdDataScrapper::doc}}\index{module@\spxentry{module}!DwdDataScrapper@\spxentry{DwdDataScrapper}}\index{DwdDataScrapper@\spxentry{DwdDataScrapper}!module@\spxentry{module}}\index{DataScrapper (class in DwdDataScrapper)@\spxentry{DataScrapper}\spxextra{class in DwdDataScrapper}}

\begin{fulllineitems}
\phantomsection\label{\detokenize{DwdDataScrapper:DwdDataScrapper.DataScrapper}}\pysiglinewithargsret{\sphinxbfcode{\sphinxupquote{class }}\sphinxcode{\sphinxupquote{DwdDataScrapper.}}\sphinxbfcode{\sphinxupquote{DataScrapper}}}{\emph{\DUrole{n}{external\_domain}}, \emph{\DUrole{n}{external\_path}}, \emph{\DUrole{n}{local\_domain}\DUrole{o}{=}\DUrole{default_value}{\textquotesingle{}\textquotesingle{}}}, \emph{\DUrole{n}{ending}\DUrole{o}{=}\DUrole{default_value}{None}}, \emph{\DUrole{n}{looking\_for\_ending}\DUrole{o}{=}\DUrole{default_value}{\textquotesingle{}\textquotesingle{}}}}{}
\sphinxAtStartPar
Bases: \sphinxcode{\sphinxupquote{object}}
\begin{quote}\begin{description}
\item[{Description}] \leavevmode
\sphinxAtStartPar
This class will download and unzip your files. Check DwdDict module first!

\end{description}\end{quote}
\index{\_\_init\_\_() (DwdDataScrapper.DataScrapper method)@\spxentry{\_\_init\_\_()}\spxextra{DwdDataScrapper.DataScrapper method}}

\begin{fulllineitems}
\phantomsection\label{\detokenize{DwdDataScrapper:DwdDataScrapper.DataScrapper.__init__}}\pysiglinewithargsret{\sphinxbfcode{\sphinxupquote{\_\_init\_\_}}}{\emph{\DUrole{n}{external\_domain}}, \emph{\DUrole{n}{external\_path}}, \emph{\DUrole{n}{local\_domain}\DUrole{o}{=}\DUrole{default_value}{\textquotesingle{}\textquotesingle{}}}, \emph{\DUrole{n}{ending}\DUrole{o}{=}\DUrole{default_value}{None}}, \emph{\DUrole{n}{looking\_for\_ending}\DUrole{o}{=}\DUrole{default_value}{\textquotesingle{}\textquotesingle{}}}}{}~\begin{quote}\begin{description}
\item[{Parameters}] \leavevmode\begin{itemize}
\item {} 
\sphinxAtStartPar
\sphinxstyleliteralstrong{\sphinxupquote{external\_domain}} \textendash{} \sphinxstylestrong{as string}. Will be set from your DwdDict.py

\item {} 
\sphinxAtStartPar
\sphinxstyleliteralstrong{\sphinxupquote{external\_path}} \textendash{} \sphinxstylestrong{as string}. Will be set from your DwdDict.py

\item {} 
\sphinxAtStartPar
\sphinxstyleliteralstrong{\sphinxupquote{local\_domain}} \textendash{} \sphinxstylestrong{as string}. Tells where to save your downloading and unzipping

\item {} 
\sphinxAtStartPar
\sphinxstyleliteralstrong{\sphinxupquote{ending}} \textendash{} \sphinxstylestrong{as list}. Will be set from your DwdDict.py

\item {} 
\sphinxAtStartPar
\sphinxstyleliteralstrong{\sphinxupquote{looking\_for\_ending}} \textendash{} \sphinxstylestrong{as string}. Skip this (not important for now, maybe in future)

\end{itemize}

\end{description}\end{quote}

\end{fulllineitems}

\index{download() (DwdDataScrapper.DataScrapper method)@\spxentry{download()}\spxextra{DwdDataScrapper.DataScrapper method}}

\begin{fulllineitems}
\phantomsection\label{\detokenize{DwdDataScrapper:DwdDataScrapper.DataScrapper.download}}\pysiglinewithargsret{\sphinxbfcode{\sphinxupquote{download}}}{\emph{\DUrole{n}{download\_path}}}{}
\sphinxAtStartPar
\sphinxstylestrong{Description:} Is the main downloading method. It will download the data for you. It will also create folders of the same structure,
that the path is in your local\_domain (if not existing already). After the download it will create the needed .json file in the folder,
where the data was saved. It will be called “last\_download\_date” and it will include some information about the filepath download\sphinxhyphen{}date and filename.
\begin{quote}\begin{description}
\item[{Parameters}] \leavevmode
\sphinxAtStartPar
\sphinxstyleliteralstrong{\sphinxupquote{download\_path}} \textendash{} \sphinxstylestrong{as string}. Can be generated by self.external\_directory\_indicator(). Or if you want to download something manually. The path would look like this: “climate\_environment/CDC/observations\_germany/climate/10\_minutes/air\_temperature/historical/”

\item[{Returns}] \leavevmode
\sphinxAtStartPar
“download finished” if succeeded.

\end{description}\end{quote}

\end{fulllineitems}

\index{download\_loop() (DwdDataScrapper.DataScrapper method)@\spxentry{download\_loop()}\spxextra{DwdDataScrapper.DataScrapper method}}

\begin{fulllineitems}
\phantomsection\label{\detokenize{DwdDataScrapper:DwdDataScrapper.DataScrapper.download_loop}}\pysiglinewithargsret{\sphinxbfcode{\sphinxupquote{download\_loop}}}{}{}
\sphinxAtStartPar
\sphinxstylestrong{Description:} Creates the download loop. Will download all the data from the self.external\_directory\_indicator()
\begin{quote}\begin{description}
\item[{Returns}] \leavevmode
\sphinxAtStartPar
None

\end{description}\end{quote}

\end{fulllineitems}

\index{external\_directory\_indicator() (DwdDataScrapper.DataScrapper method)@\spxentry{external\_directory\_indicator()}\spxextra{DwdDataScrapper.DataScrapper method}}

\begin{fulllineitems}
\phantomsection\label{\detokenize{DwdDataScrapper:DwdDataScrapper.DataScrapper.external_directory_indicator}}\pysiglinewithargsret{\sphinxbfcode{\sphinxupquote{external\_directory\_indicator}}}{\emph{\DUrole{n}{pre\_extend\_list}\DUrole{o}{=}\DUrole{default_value}{\textquotesingle{}\textquotesingle{}}}}{}
\sphinxAtStartPar
\sphinxstylestrong{Description:} Will generate a list (verzeichnis\_list) of oll the “href” (links and urls) on a given page
\begin{quote}\begin{description}
\item[{Parameters}] \leavevmode
\sphinxAtStartPar
\sphinxstyleliteralstrong{\sphinxupquote{pre\_extend\_list}} \textendash{} \sphinxstylestrong{as string}. Skip this (not important for now, maybe in future)

\item[{Returns}] \leavevmode
\sphinxAtStartPar
verzeichnis\_list \sphinxstylestrong{as list}

\end{description}\end{quote}

\end{fulllineitems}

\index{extract\_zip() (DwdDataScrapper.DataScrapper method)@\spxentry{extract\_zip()}\spxextra{DwdDataScrapper.DataScrapper method}}

\begin{fulllineitems}
\phantomsection\label{\detokenize{DwdDataScrapper:DwdDataScrapper.DataScrapper.extract_zip}}\pysiglinewithargsret{\sphinxbfcode{\sphinxupquote{extract\_zip}}}{\emph{\DUrole{n}{looking\_for}\DUrole{o}{=}\DUrole{default_value}{\textquotesingle{}.zip\textquotesingle{}}}}{}
\sphinxAtStartPar
\sphinxstylestrong{Description:} Extract all the files inside a path
\begin{quote}\begin{description}
\item[{Parameters}] \leavevmode
\sphinxAtStartPar
\sphinxstyleliteralstrong{\sphinxupquote{looking\_for}} \textendash{} \sphinxstylestrong{as string}. Should be “.zip”.

\item[{Returns}] \leavevmode
\sphinxAtStartPar
“Extracting finished” if succeeded.

\end{description}\end{quote}

\end{fulllineitems}

\index{generate\_local\_list() (DwdDataScrapper.DataScrapper method)@\spxentry{generate\_local\_list()}\spxextra{DwdDataScrapper.DataScrapper method}}

\begin{fulllineitems}
\phantomsection\label{\detokenize{DwdDataScrapper:DwdDataScrapper.DataScrapper.generate_local_list}}\pysiglinewithargsret{\sphinxbfcode{\sphinxupquote{generate\_local\_list}}}{}{}
\sphinxAtStartPar
\sphinxstylestrong{Description:} Will generate a local\_list with all filenames.
\begin{quote}\begin{description}
\item[{Returns}] \leavevmode
\sphinxAtStartPar
local\_list \sphinxstylestrong{as list}

\end{description}\end{quote}

\end{fulllineitems}

\index{get\_external\_date() (DwdDataScrapper.DataScrapper method)@\spxentry{get\_external\_date()}\spxextra{DwdDataScrapper.DataScrapper method}}

\begin{fulllineitems}
\phantomsection\label{\detokenize{DwdDataScrapper:DwdDataScrapper.DataScrapper.get_external_date}}\pysiglinewithargsret{\sphinxbfcode{\sphinxupquote{get\_external\_date}}}{}{}
\sphinxAtStartPar
\sphinxstylestrong{Description:} Will generate a list (text\_elemnts) of all text elements on a page (here the date) (will be important in future).
\begin{quote}\begin{description}
\item[{Returns}] \leavevmode
\sphinxAtStartPar
text\_elements \sphinxstylestrong{as list}

\end{description}\end{quote}

\end{fulllineitems}

\index{main\_update\_data() (DwdDataScrapper.DataScrapper method)@\spxentry{main\_update\_data()}\spxextra{DwdDataScrapper.DataScrapper method}}

\begin{fulllineitems}
\phantomsection\label{\detokenize{DwdDataScrapper:DwdDataScrapper.DataScrapper.main_update_data}}\pysiglinewithargsret{\sphinxbfcode{\sphinxupquote{main\_update\_data}}}{}{}
\sphinxAtStartPar
\sphinxstylestrong{Description:} Compressed method for the download. Less complicated.
\begin{quote}\begin{description}
\item[{Returns}] \leavevmode
\sphinxAtStartPar
“Data updated” if succeeded.

\end{description}\end{quote}

\end{fulllineitems}


\end{fulllineitems}



\subsection{getting started: download all (air\_temperatur, solar, wind, precipitation)}
\label{\detokenize{DwdDataScrapper:getting-started-download-all-air-temperatur-solar-wind-precipitation}}
\begin{sphinxVerbatim}[commandchars=\\\{\}]
\PYG{k+kn}{from} \PYG{n+nn}{DwdMain} \PYG{k+kn}{import} \PYG{n}{main\PYGZus{}dwd}
\PYG{n}{main\PYGZus{}dwd}\PYG{p}{(}\PYG{n}{local\PYGZus{}domain}\PYG{o}{=}\PYG{l+s+s2}{\PYGZdq{}}\PYG{l+s+s2}{YOUR\PYGZus{}PATH/}\PYG{l+s+s2}{\PYGZdq{}}\PYG{p}{)}\PYG{o}{.}\PYG{n}{main\PYGZus{}datascrapper}\PYG{p}{(}\PYG{n+nb}{all}\PYG{o}{=}\PYG{k+kc}{True}\PYG{p}{)}
\PYG{c+c1}{\PYGZsh{} Will download and unzip all the data for air\PYGZus{}temperatur, solar, wind, precipitation (historical, meta\PYGZus{}data, now and recent) to YOUR\PYGZus{}PATH/.}
\PYG{n}{main\PYGZus{}dwd}\PYG{p}{(}\PYG{n}{local\PYGZus{}domain}\PYG{o}{=}\PYG{l+s+s2}{\PYGZdq{}}\PYG{l+s+s2}{YOUR\PYGZus{}PATH/}\PYG{l+s+s2}{\PYGZdq{}}\PYG{p}{)}\PYG{o}{.}\PYG{n}{main\PYGZus{}writer}\PYG{p}{(}\PYG{n+nb}{all}\PYG{o}{=}\PYG{k+kc}{True}\PYG{p}{)}
\PYG{c+c1}{\PYGZsh{} Will create some .json files inside .../extracted\PYGZus{}files/ for faster loading times. This step is important. Inside this .json files will dates and paths for every station.}
\end{sphinxVerbatim}


\section{DwdDict module}
\label{\detokenize{DwdDict:module-DwdDict}}\label{\detokenize{DwdDict:dwddict-module}}\label{\detokenize{DwdDict::doc}}\index{module@\spxentry{module}!DwdDict@\spxentry{DwdDict}}\index{DwdDict@\spxentry{DwdDict}!module@\spxentry{module}}\begin{quote}\begin{description}
\item[{Describtion}] \leavevmode
\sphinxAtStartPar
Is a global dictionary for all data you are scraping from DWD/CDC. If you want to change from 10\_minutes resolution to 1\_minute resolution of your data, you might start here with few changes.

\end{description}\end{quote}


\begin{savenotes}\sphinxattablestart
\centering
\begin{tabulary}{\linewidth}[t]{|T|T|T|}
\hline
\sphinxstyletheadfamily 
\sphinxAtStartPar
Name
&\sphinxstyletheadfamily 
\sphinxAtStartPar
type
&\sphinxstyletheadfamily 
\sphinxAtStartPar
description
\\
\hline
\sphinxAtStartPar
external\_domain
&
\sphinxAtStartPar
str
&
\sphinxAtStartPar
What \sphinxstylestrong{domain} do you use for your data?
\\
\hline
\sphinxAtStartPar
external\_global\_path
&
\sphinxAtStartPar
str
&
\sphinxAtStartPar
What is the data path for your external\_domain?

\sphinxAtStartPar
\sphinxstylestrong{Note:} Here you could change for 1\_minute.
\\
\hline
\sphinxAtStartPar
ending
&
\sphinxAtStartPar
list
&
\sphinxAtStartPar
Is important for for the \sphinxstylestrong{download}.

\sphinxAtStartPar
It will download all the data with this given endings.
\\
\hline
\sphinxAtStartPar
type\_dict
&
\sphinxAtStartPar
dict
&
\sphinxAtStartPar
Describes what the \sphinxstylestrong{prefix} is before any station\_id.

\sphinxAtStartPar
Example: You downloaded data for solar. If you check

\sphinxAtStartPar
the file names which you downloaded. It should be something

\sphinxAtStartPar
like: 10minutenwerte\_SOLAR\_00003\_19930428\_19991231\_hist.

\sphinxAtStartPar
So the prefix would be \sphinxstylestrong{SOLAR}.
\\
\hline
\sphinxAtStartPar
load\_txt\_dict
&
\sphinxAtStartPar
dict
&
\sphinxAtStartPar
Describes what the \sphinxstylestrong{prefix} (for given station\_list) is.

\sphinxAtStartPar
DWD gives you a simple .txt file, that got some different

\sphinxAtStartPar
names and this .txt describes what stations are given for the type\_of\_data.

\sphinxAtStartPar
Example: You downloaded data for solar. if you check the

\sphinxAtStartPar
folder there should something like: zehn\_min\_sd\_Beschreibung\_Stationen.

\sphinxAtStartPar
So the prefix would be \sphinxstylestrong{sd}
\\
\hline
\sphinxAtStartPar
rest\_dict
&
\sphinxAtStartPar
dict
&
\sphinxAtStartPar
Since all the given .txt are not completely uniform. We need this dict.

\sphinxAtStartPar
It describes the part of the given .txt names.

\sphinxAtStartPar
Example: zehn\_now\_sd\_Beschreibung\_Stationen and

\sphinxAtStartPar
zehn\_min\_sd\_Beschreibung\_Stationen are different.

\sphinxAtStartPar
There is a \sphinxstylestrong{now} and a \sphinxstylestrong{min}.
\\
\hline
\sphinxAtStartPar
title\_dict
&
\sphinxAtStartPar
dict
&
\sphinxAtStartPar
Is important for \sphinxstylestrong{plotting} your data. It will be the \sphinxstylestrong{title} of your graphs.
\\
\hline
\sphinxAtStartPar
unit\_dict
&
\sphinxAtStartPar
dict
&
\sphinxAtStartPar
Is important for \sphinxstylestrong{plotting} your data. This units will be on your \sphinxstylestrong{axes}.
\\
\hline
\sphinxAtStartPar
type\_of\_time\_list
&
\sphinxAtStartPar
list
&
\sphinxAtStartPar
Describes the types of time, that are given. Is important for \sphinxstylestrong{writing}.
\\
\hline
\sphinxAtStartPar
type\_of\_data\_list
&
\sphinxAtStartPar
list
&
\sphinxAtStartPar
Describes the types of data, that are given. Is important for \sphinxstylestrong{downloading}.
\\
\hline
\end{tabulary}
\par
\sphinxattableend\end{savenotes}
\index{get\_dwd\_dict() (in module DwdDict)@\spxentry{get\_dwd\_dict()}\spxextra{in module DwdDict}}

\begin{fulllineitems}
\phantomsection\label{\detokenize{DwdDict:DwdDict.get_dwd_dict}}\pysiglinewithargsret{\sphinxcode{\sphinxupquote{DwdDict.}}\sphinxbfcode{\sphinxupquote{get\_dwd\_dict}}}{}{}~\begin{quote}\begin{description}
\item[{Returns}] \leavevmode
\sphinxAtStartPar
type\_dict, load\_txt\_dict, rest\_dict, title\_dict, unit\_dict, type\_of\_time\_list, type\_of\_data\_list, external\_domain, external\_path\_global, ending

\end{description}\end{quote}

\end{fulllineitems}



\section{DwdMain module}
\label{\detokenize{DwdMain:module-DwdMain}}\label{\detokenize{DwdMain:dwdmain-module}}\label{\detokenize{DwdMain::doc}}\index{module@\spxentry{module}!DwdMain@\spxentry{DwdMain}}\index{DwdMain@\spxentry{DwdMain}!module@\spxentry{module}}\index{DwdMain (class in DwdMain)@\spxentry{DwdMain}\spxextra{class in DwdMain}}

\begin{fulllineitems}
\phantomsection\label{\detokenize{DwdMain:DwdMain.DwdMain}}\pysiglinewithargsret{\sphinxbfcode{\sphinxupquote{class }}\sphinxcode{\sphinxupquote{DwdMain.}}\sphinxbfcode{\sphinxupquote{DwdMain}}}{\emph{\DUrole{n}{external\_domain}}, \emph{\DUrole{n}{external\_path\_global}}, \emph{\DUrole{n}{local\_domain}}, \emph{\DUrole{n}{type\_of\_data}}, \emph{\DUrole{n}{type\_of\_time}}, \emph{\DUrole{n}{type\_dict}}, \emph{\DUrole{n}{load\_txt\_dict}}, \emph{\DUrole{n}{rest\_dict}}, \emph{\DUrole{n}{ending}\DUrole{o}{=}\DUrole{default_value}{None}}, \emph{\DUrole{n}{start\_date}\DUrole{o}{=}\DUrole{default_value}{None}}, \emph{\DUrole{n}{end\_date}\DUrole{o}{=}\DUrole{default_value}{None}}, \emph{\DUrole{n}{compare\_station}\DUrole{o}{=}\DUrole{default_value}{None}}, \emph{\DUrole{n}{x\_coordinate}\DUrole{o}{=}\DUrole{default_value}{None}}, \emph{\DUrole{n}{y\_coordinate}\DUrole{o}{=}\DUrole{default_value}{None}}, \emph{\DUrole{n}{z\_coordinate}\DUrole{o}{=}\DUrole{default_value}{None}}, \emph{\DUrole{n}{k\_factor}\DUrole{o}{=}\DUrole{default_value}{None}}, \emph{\DUrole{n}{looking\_for}\DUrole{o}{=}\DUrole{default_value}{None}}, \emph{\DUrole{n}{type\_of\_time\_list}\DUrole{o}{=}\DUrole{default_value}{None}}, \emph{\DUrole{n}{type\_of\_data\_list}\DUrole{o}{=}\DUrole{default_value}{None}}, \emph{\DUrole{n}{unit\_dict}\DUrole{o}{=}\DUrole{default_value}{None}}, \emph{\DUrole{n}{title\_dict}\DUrole{o}{=}\DUrole{default_value}{None}}}{}
\sphinxAtStartPar
Bases: \sphinxcode{\sphinxupquote{object}}
\begin{quote}\begin{description}
\item[{Description}] \leavevmode
\sphinxAtStartPar
Compressing all the modules into one module for easier use.

\end{description}\end{quote}
\index{\_\_init\_\_() (DwdMain.DwdMain method)@\spxentry{\_\_init\_\_()}\spxextra{DwdMain.DwdMain method}}

\begin{fulllineitems}
\phantomsection\label{\detokenize{DwdMain:DwdMain.DwdMain.__init__}}\pysiglinewithargsret{\sphinxbfcode{\sphinxupquote{\_\_init\_\_}}}{\emph{\DUrole{n}{external\_domain}}, \emph{\DUrole{n}{external\_path\_global}}, \emph{\DUrole{n}{local\_domain}}, \emph{\DUrole{n}{type\_of\_data}}, \emph{\DUrole{n}{type\_of\_time}}, \emph{\DUrole{n}{type\_dict}}, \emph{\DUrole{n}{load\_txt\_dict}}, \emph{\DUrole{n}{rest\_dict}}, \emph{\DUrole{n}{ending}\DUrole{o}{=}\DUrole{default_value}{None}}, \emph{\DUrole{n}{start\_date}\DUrole{o}{=}\DUrole{default_value}{None}}, \emph{\DUrole{n}{end\_date}\DUrole{o}{=}\DUrole{default_value}{None}}, \emph{\DUrole{n}{compare\_station}\DUrole{o}{=}\DUrole{default_value}{None}}, \emph{\DUrole{n}{x\_coordinate}\DUrole{o}{=}\DUrole{default_value}{None}}, \emph{\DUrole{n}{y\_coordinate}\DUrole{o}{=}\DUrole{default_value}{None}}, \emph{\DUrole{n}{z\_coordinate}\DUrole{o}{=}\DUrole{default_value}{None}}, \emph{\DUrole{n}{k\_factor}\DUrole{o}{=}\DUrole{default_value}{None}}, \emph{\DUrole{n}{looking\_for}\DUrole{o}{=}\DUrole{default_value}{None}}, \emph{\DUrole{n}{type\_of\_time\_list}\DUrole{o}{=}\DUrole{default_value}{None}}, \emph{\DUrole{n}{type\_of\_data\_list}\DUrole{o}{=}\DUrole{default_value}{None}}, \emph{\DUrole{n}{unit\_dict}\DUrole{o}{=}\DUrole{default_value}{None}}, \emph{\DUrole{n}{title\_dict}\DUrole{o}{=}\DUrole{default_value}{None}}}{}~\begin{quote}\begin{description}
\item[{Parameters}] \leavevmode\begin{itemize}
\item {} 
\sphinxAtStartPar
\sphinxstyleliteralstrong{\sphinxupquote{external\_domain}} \textendash{} \sphinxstylestrong{as string}. Choose the domain.

\item {} 
\sphinxAtStartPar
\sphinxstyleliteralstrong{\sphinxupquote{external\_path\_global}} \textendash{} \sphinxstylestrong{as string}.  Choose the path inside this domain.

\item {} 
\sphinxAtStartPar
\sphinxstyleliteralstrong{\sphinxupquote{local\_domain}} \textendash{} \sphinxstylestrong{as string}. Choose your local domain.

\item {} 
\sphinxAtStartPar
\sphinxstyleliteralstrong{\sphinxupquote{type\_of\_data}} \textendash{} \sphinxstylestrong{as string}. Choose the type of data

\item {} 
\sphinxAtStartPar
\sphinxstyleliteralstrong{\sphinxupquote{type\_of\_time}} \textendash{} \sphinxstylestrong{as string}. Choose the type of time

\item {} 
\sphinxAtStartPar
\sphinxstyleliteralstrong{\sphinxupquote{type\_dict}} \textendash{} \sphinxstylestrong{as dict}. Should be the dictionary from DwdDict.py type\_dict.

\item {} 
\sphinxAtStartPar
\sphinxstyleliteralstrong{\sphinxupquote{load\_txt\_dict}} \textendash{} \sphinxstylestrong{as dict}. Should be the dictionary from DwdDict.py load\_txt\_dict.

\item {} 
\sphinxAtStartPar
\sphinxstyleliteralstrong{\sphinxupquote{rest\_dict}} \textendash{} \sphinxstylestrong{as dict}. Should be the dictionary from DwdDict.py rest\_dict.

\item {} 
\sphinxAtStartPar
\sphinxstyleliteralstrong{\sphinxupquote{ending}} \textendash{} \sphinxstylestrong{as list}. Should be the list from DwdDict.py ending.

\item {} 
\sphinxAtStartPar
\sphinxstyleliteralstrong{\sphinxupquote{start\_date}} \textendash{} \sphinxstylestrong{as int}. YYYYMMDDHHMM. Your start date.

\item {} 
\sphinxAtStartPar
\sphinxstyleliteralstrong{\sphinxupquote{end\_date}} \textendash{} \sphinxstylestrong{as int}. YYYYMMDDHHMM. Your end date.

\item {} 
\sphinxAtStartPar
\sphinxstyleliteralstrong{\sphinxupquote{compare\_station}} \textendash{} \sphinxstylestrong{as string}. Only if you want to compare your calculation with a station.

\item {} 
\sphinxAtStartPar
\sphinxstyleliteralstrong{\sphinxupquote{x\_coordinate}} \textendash{} \sphinxstylestrong{as float}. X\sphinxhyphen{}coordinates for your chosen location.

\item {} 
\sphinxAtStartPar
\sphinxstyleliteralstrong{\sphinxupquote{y\_coordinate}} \textendash{} \sphinxstylestrong{as float}. Y\sphinxhyphen{}coordinates for your chosen location.

\item {} 
\sphinxAtStartPar
\sphinxstyleliteralstrong{\sphinxupquote{z\_coordinate}} \textendash{} \sphinxstylestrong{as float}. Z\sphinxhyphen{}coordinates for your chosen location. (Can be skipped for now, maybe important in the future)

\item {} 
\sphinxAtStartPar
\sphinxstyleliteralstrong{\sphinxupquote{k\_factor}} \textendash{} \sphinxstylestrong{as int}. How many station are you looking around your location?

\item {} 
\sphinxAtStartPar
\sphinxstyleliteralstrong{\sphinxupquote{looking\_for}} \textendash{} \sphinxstylestrong{as list}. What data you want to plot?

\item {} 
\sphinxAtStartPar
\sphinxstyleliteralstrong{\sphinxupquote{type\_of\_time\_list}} \textendash{} \sphinxstylestrong{as list}. Should be from DwdDict.py type\_of\_time\_list.

\item {} 
\sphinxAtStartPar
\sphinxstyleliteralstrong{\sphinxupquote{type\_of\_data\_list}} \textendash{} \sphinxstylestrong{as list}. Should be from DwdDict.py type\_of\_data\_list.

\item {} 
\sphinxAtStartPar
\sphinxstyleliteralstrong{\sphinxupquote{unit\_dict}} \textendash{} \sphinxstylestrong{as dict}. Should be from DwdDict.py unit\_dict.

\item {} 
\sphinxAtStartPar
\sphinxstyleliteralstrong{\sphinxupquote{title\_dict}} \textendash{} \sphinxstylestrong{as dict}. Should be from DwdDict.py title\_dict.

\end{itemize}

\end{description}\end{quote}

\end{fulllineitems}

\index{main\_activ\_stations\_in\_date() (DwdMain.DwdMain method)@\spxentry{main\_activ\_stations\_in\_date()}\spxextra{DwdMain.DwdMain method}}

\begin{fulllineitems}
\phantomsection\label{\detokenize{DwdMain:DwdMain.DwdMain.main_activ_stations_in_date}}\pysiglinewithargsret{\sphinxbfcode{\sphinxupquote{main\_activ\_stations\_in\_date}}}{}{}~\begin{quote}\begin{description}
\item[{Description}] \leavevmode
\sphinxAtStartPar
Will return all the stations for your data type and your time type and your timedelta.

\item[{Returns}] \leavevmode
\sphinxAtStartPar
x\_coordinates, y\_coordinate, station\_ids: \sphinxstylestrong{all as array, station\_ids as list}

\end{description}\end{quote}

\end{fulllineitems}

\index{main\_data\_map() (DwdMain.DwdMain method)@\spxentry{main\_data\_map()}\spxextra{DwdMain.DwdMain method}}

\begin{fulllineitems}
\phantomsection\label{\detokenize{DwdMain:DwdMain.DwdMain.main_data_map}}\pysiglinewithargsret{\sphinxbfcode{\sphinxupquote{main\_data\_map}}}{}{}~\begin{quote}\begin{description}
\item[{Description}] \leavevmode
\sphinxAtStartPar
Will create some json files, with active stations, not activ stations, activ stations without data and the near station for your location and your timedelta

\item[{Returns}] \leavevmode
\sphinxAtStartPar
“zipped\_data\_for\_active\_map” if succeeded.

\end{description}\end{quote}

\end{fulllineitems}

\index{main\_datascrapper() (DwdMain.DwdMain method)@\spxentry{main\_datascrapper()}\spxextra{DwdMain.DwdMain method}}

\begin{fulllineitems}
\phantomsection\label{\detokenize{DwdMain:DwdMain.DwdMain.main_datascrapper}}\pysiglinewithargsret{\sphinxbfcode{\sphinxupquote{main\_datascrapper}}}{\emph{\DUrole{n}{all}\DUrole{o}{=}\DUrole{default_value}{False}}}{}~\begin{quote}\begin{description}
\item[{Description}] \leavevmode
\sphinxAtStartPar
Is the main\sphinxhyphen{}methode for DataScrapper.py the data. Check the DwdDataScrapper module for more information

\item[{Parameters}] \leavevmode
\sphinxAtStartPar
\sphinxstyleliteralstrong{\sphinxupquote{all}} \textendash{} \sphinxstylestrong{as Boolean}. If all==True, it will download all the data from DwdDict.py type\_of\_data\_list.

\item[{Returns}] \leavevmode
\sphinxAtStartPar
None

\end{description}\end{quote}

\end{fulllineitems}

\index{main\_plotter\_data() (DwdMain.DwdMain method)@\spxentry{main\_plotter\_data()}\spxextra{DwdMain.DwdMain method}}

\begin{fulllineitems}
\phantomsection\label{\detokenize{DwdMain:DwdMain.DwdMain.main_plotter_data}}\pysiglinewithargsret{\sphinxbfcode{\sphinxupquote{main\_plotter\_data}}}{\emph{\DUrole{n}{qn\_weight}\DUrole{o}{=}\DUrole{default_value}{False}}, \emph{\DUrole{n}{distance\_weight}\DUrole{o}{=}\DUrole{default_value}{False}}, \emph{\DUrole{n}{compare}\DUrole{o}{=}\DUrole{default_value}{False}}, \emph{\DUrole{n}{no\_plot}\DUrole{o}{=}\DUrole{default_value}{False}}}{}~\begin{quote}\begin{description}
\item[{Description}] \leavevmode
\sphinxAtStartPar
Compressed method for plotting. qn\_weight, distance\_weight are the methods. If both are False, it will make simple average method.

\item[{Parameters}] \leavevmode\begin{itemize}
\item {} 
\sphinxAtStartPar
\sphinxstyleliteralstrong{\sphinxupquote{qn\_weight}} \textendash{} \sphinxstylestrong{as Boolean}. If True weight the quality of data.

\item {} 
\sphinxAtStartPar
\sphinxstyleliteralstrong{\sphinxupquote{distance\_weight}} \textendash{} \sphinxstylestrong{as Boolean}. If True weight the distance of data.

\item {} 
\sphinxAtStartPar
\sphinxstyleliteralstrong{\sphinxupquote{compare}} \textendash{} \sphinxstylestrong{as Boolean}. If you want to compare, how accurate your methods are, set True and choose coordinates for a station.

\item {} 
\sphinxAtStartPar
\sphinxstyleliteralstrong{\sphinxupquote{no\_plot}} \textendash{} \sphinxstylestrong{as Boolean}. If you want to see just the result, without the plot (saves a bit of time), you can set this True.

\end{itemize}

\item[{Returns}] \leavevmode
\sphinxAtStartPar
calculations and graphs (“plot saved”)

\end{description}\end{quote}

\end{fulllineitems}

\index{main\_plotter\_stations() (DwdMain.DwdMain method)@\spxentry{main\_plotter\_stations()}\spxextra{DwdMain.DwdMain method}}

\begin{fulllineitems}
\phantomsection\label{\detokenize{DwdMain:DwdMain.DwdMain.main_plotter_stations}}\pysiglinewithargsret{\sphinxbfcode{\sphinxupquote{main\_plotter\_stations}}}{\emph{\DUrole{n}{projection}\DUrole{o}{=}\DUrole{default_value}{False}}}{}~\begin{quote}\begin{description}
\item[{Description}] \leavevmode
\sphinxAtStartPar
Calling the method from DwdPlotter.PlotterForStations(…).plotting\_3d(…) and .plotting\_height\_2d()

\item[{Parameters}] \leavevmode
\sphinxAtStartPar
\sphinxstyleliteralstrong{\sphinxupquote{projection}} \textendash{} \sphinxstylestrong{as Boolean}. If projection==True, it will project the height of the stations.

\item[{Returns}] \leavevmode
\sphinxAtStartPar
two times “plot saved” if succeeded.

\end{description}\end{quote}

\end{fulllineitems}

\index{main\_station\_array() (DwdMain.DwdMain method)@\spxentry{main\_station\_array()}\spxextra{DwdMain.DwdMain method}}

\begin{fulllineitems}
\phantomsection\label{\detokenize{DwdMain:DwdMain.DwdMain.main_station_array}}\pysiglinewithargsret{\sphinxbfcode{\sphinxupquote{main\_station\_array}}}{}{}~\begin{quote}\begin{description}
\item[{Description}] \leavevmode
\sphinxAtStartPar
Will return all the stations for your data type and your time type.

\end{description}\end{quote}

\sphinxAtStartPar
:return:\sphinxstylestrong{as array}

\end{fulllineitems}

\index{main\_station\_information() (DwdMain.DwdMain method)@\spxentry{main\_station\_information()}\spxextra{DwdMain.DwdMain method}}

\begin{fulllineitems}
\phantomsection\label{\detokenize{DwdMain:DwdMain.DwdMain.main_station_information}}\pysiglinewithargsret{\sphinxbfcode{\sphinxupquote{main\_station\_information}}}{\emph{\DUrole{n}{station\_id}}}{}~\begin{quote}\begin{description}
\item[{Description}] \leavevmode
\sphinxAtStartPar
Will return a dict for a station with all the important informations

\item[{Parameters}] \leavevmode
\sphinxAtStartPar
\sphinxstyleliteralstrong{\sphinxupquote{station\_id}} \textendash{} \sphinxstylestrong{as string}. The stationid with prefix.

\item[{Returns}] \leavevmode
\sphinxAtStartPar
\sphinxstylestrong{as dict}

\end{description}\end{quote}

\end{fulllineitems}

\index{main\_writer() (DwdMain.DwdMain method)@\spxentry{main\_writer()}\spxextra{DwdMain.DwdMain method}}

\begin{fulllineitems}
\phantomsection\label{\detokenize{DwdMain:DwdMain.DwdMain.main_writer}}\pysiglinewithargsret{\sphinxbfcode{\sphinxupquote{main\_writer}}}{\emph{\DUrole{n}{type\_of\_data\_list}\DUrole{o}{=}\DUrole{default_value}{None}}, \emph{\DUrole{n}{type\_of\_time\_list}\DUrole{o}{=}\DUrole{default_value}{None}}, \emph{\DUrole{n}{all}\DUrole{o}{=}\DUrole{default_value}{False}}}{}~\begin{quote}\begin{description}
\item[{Description}] \leavevmode
\sphinxAtStartPar
Is the main\sphinxhyphen{}methode for DwdDataPrep.py class Writer. Check the DwdDataPrep module for more information

\item[{Parameters}] \leavevmode\begin{itemize}
\item {} 
\sphinxAtStartPar
\sphinxstyleliteralstrong{\sphinxupquote{type\_of\_data\_list}} \textendash{} \sphinxstylestrong{as list}.

\item {} 
\sphinxAtStartPar
\sphinxstyleliteralstrong{\sphinxupquote{type\_of\_time\_list}} \textendash{} \sphinxstylestrong{as list}.

\item {} 
\sphinxAtStartPar
\sphinxstyleliteralstrong{\sphinxupquote{all}} \textendash{} \sphinxstylestrong{as Boolean}. If all==True, it will write all the data from DwdDict.py type\_of\_data\_list and type\_of\_time\_list. Else you can give your own lists.

\end{itemize}

\item[{Returns}] \leavevmode
\sphinxAtStartPar
“data written” if succeeded.

\end{description}\end{quote}

\end{fulllineitems}


\end{fulllineitems}

\index{main\_dwd() (in module DwdMain)@\spxentry{main\_dwd()}\spxextra{in module DwdMain}}

\begin{fulllineitems}
\phantomsection\label{\detokenize{DwdMain:DwdMain.main_dwd}}\pysiglinewithargsret{\sphinxcode{\sphinxupquote{DwdMain.}}\sphinxbfcode{\sphinxupquote{main\_dwd}}}{\emph{\DUrole{n}{local\_domain}}, \emph{\DUrole{n}{type\_of\_data}\DUrole{o}{=}\DUrole{default_value}{None}}, \emph{\DUrole{n}{type\_of\_time}\DUrole{o}{=}\DUrole{default_value}{None}}, \emph{\DUrole{n}{start\_date}\DUrole{o}{=}\DUrole{default_value}{None}}, \emph{\DUrole{n}{end\_date}\DUrole{o}{=}\DUrole{default_value}{None}}, \emph{\DUrole{n}{compare\_station}\DUrole{o}{=}\DUrole{default_value}{None}}, \emph{\DUrole{n}{x\_coordinate}\DUrole{o}{=}\DUrole{default_value}{None}}, \emph{\DUrole{n}{y\_coordinate}\DUrole{o}{=}\DUrole{default_value}{None}}, \emph{\DUrole{n}{z\_coordinate}\DUrole{o}{=}\DUrole{default_value}{None}}, \emph{\DUrole{n}{k\_factor}\DUrole{o}{=}\DUrole{default_value}{None}}, \emph{\DUrole{n}{looking\_for}\DUrole{o}{=}\DUrole{default_value}{None}}}{}~\begin{quote}\begin{description}
\item[{Description}] \leavevmode
\sphinxAtStartPar
To make your life easier, this grabs all the important informations from DwdDict automatically.

\item[{Parameters}] \leavevmode\begin{itemize}
\item {} 
\sphinxAtStartPar
\sphinxstyleliteralstrong{\sphinxupquote{local\_domain}} \textendash{} \sphinxstylestrong{as string}. Choose your local domain.

\item {} 
\sphinxAtStartPar
\sphinxstyleliteralstrong{\sphinxupquote{type\_of\_data}} \textendash{} \sphinxstylestrong{as string}. Choose the type of data

\item {} 
\sphinxAtStartPar
\sphinxstyleliteralstrong{\sphinxupquote{type\_of\_time}} \textendash{} \sphinxstylestrong{as string}. Choose the type of time

\item {} 
\sphinxAtStartPar
\sphinxstyleliteralstrong{\sphinxupquote{start\_date}} \textendash{} \sphinxstylestrong{as int}. YYYYMMDDHHMM. Your start date.

\item {} 
\sphinxAtStartPar
\sphinxstyleliteralstrong{\sphinxupquote{end\_date}} \textendash{} \sphinxstylestrong{as int}. YYYYMMDDHHMM. Your end date.

\item {} 
\sphinxAtStartPar
\sphinxstyleliteralstrong{\sphinxupquote{compare\_station}} \textendash{} \sphinxstylestrong{as string}. Only if you want to compare your calculation with a station.

\item {} 
\sphinxAtStartPar
\sphinxstyleliteralstrong{\sphinxupquote{x\_coordinate}} \textendash{} \sphinxstylestrong{as float}. X\sphinxhyphen{}coordinates for your chosen location.

\item {} 
\sphinxAtStartPar
\sphinxstyleliteralstrong{\sphinxupquote{y\_coordinate}} \textendash{} \sphinxstylestrong{as float}. Y\sphinxhyphen{}coordinates for your chosen location.

\item {} 
\sphinxAtStartPar
\sphinxstyleliteralstrong{\sphinxupquote{z\_coordinate}} \textendash{} \sphinxstylestrong{as float}. Z\sphinxhyphen{}coordinates for your chosen location. (Can be skipped for now, maybe important in the future)

\item {} 
\sphinxAtStartPar
\sphinxstyleliteralstrong{\sphinxupquote{k\_factor}} \textendash{} \sphinxstylestrong{as int}. How many station are you looking around your location?

\item {} 
\sphinxAtStartPar
\sphinxstyleliteralstrong{\sphinxupquote{looking\_for}} \textendash{} \sphinxstylestrong{as list}. What data you want to plot?

\end{itemize}

\item[{Returns}] \leavevmode
\sphinxAtStartPar
dwd \sphinxstylestrong{as object}

\end{description}\end{quote}

\end{fulllineitems}



\section{DwdNearNeighbor module}
\label{\detokenize{DwdNearNeighbor:module-DwdNearNeighbor}}\label{\detokenize{DwdNearNeighbor:dwdnearneighbor-module}}\label{\detokenize{DwdNearNeighbor::doc}}\index{module@\spxentry{module}!DwdNearNeighbor@\spxentry{DwdNearNeighbor}}\index{DwdNearNeighbor@\spxentry{DwdNearNeighbor}!module@\spxentry{module}}\index{NearNeighbor (class in DwdNearNeighbor)@\spxentry{NearNeighbor}\spxextra{class in DwdNearNeighbor}}

\begin{fulllineitems}
\phantomsection\label{\detokenize{DwdNearNeighbor:DwdNearNeighbor.NearNeighbor}}\pysiglinewithargsret{\sphinxbfcode{\sphinxupquote{class }}\sphinxcode{\sphinxupquote{DwdNearNeighbor.}}\sphinxbfcode{\sphinxupquote{NearNeighbor}}}{\emph{\DUrole{n}{x\_active}}, \emph{\DUrole{n}{y\_active}}, \emph{\DUrole{n}{z\_active}}, \emph{\DUrole{n}{zip\_data\_active}}, \emph{\DUrole{n}{k\_factor}}, \emph{\DUrole{n}{start\_date}}, \emph{\DUrole{n}{end\_date}}, \emph{\DUrole{n}{activ\_id}}, \emph{\DUrole{n}{station\_list}}}{}
\sphinxAtStartPar
Bases: \sphinxcode{\sphinxupquote{object}}
\begin{quote}\begin{description}
\item[{Description}] \leavevmode
\sphinxAtStartPar
This class is finding near neighbors for the location you choosed and creates all the needed data as a DataFrame

\end{description}\end{quote}
\index{\_\_init\_\_() (DwdNearNeighbor.NearNeighbor method)@\spxentry{\_\_init\_\_()}\spxextra{DwdNearNeighbor.NearNeighbor method}}

\begin{fulllineitems}
\phantomsection\label{\detokenize{DwdNearNeighbor:DwdNearNeighbor.NearNeighbor.__init__}}\pysiglinewithargsret{\sphinxbfcode{\sphinxupquote{\_\_init\_\_}}}{\emph{\DUrole{n}{x\_active}}, \emph{\DUrole{n}{y\_active}}, \emph{\DUrole{n}{z\_active}}, \emph{\DUrole{n}{zip\_data\_active}}, \emph{\DUrole{n}{k\_factor}}, \emph{\DUrole{n}{start\_date}}, \emph{\DUrole{n}{end\_date}}, \emph{\DUrole{n}{activ\_id}}, \emph{\DUrole{n}{station\_list}}}{}~\begin{quote}\begin{description}
\item[{Parameters}] \leavevmode\begin{itemize}
\item {} 
\sphinxAtStartPar
\sphinxstyleliteralstrong{\sphinxupquote{x\_active}} \textendash{} x\sphinxhyphen{}coordinates (geoLaenge): \sphinxstylestrong{as array}. Should be set from DwdDataPrep.data\_prep\_for\_new\_location()

\item {} 
\sphinxAtStartPar
\sphinxstyleliteralstrong{\sphinxupquote{y\_active}} \textendash{} y\sphinxhyphen{}coordinates (geoBreite): \sphinxstylestrong{as array}. Should be set from DwdDataPrep.data\_prep\_for\_new\_location()

\item {} 
\sphinxAtStartPar
\sphinxstyleliteralstrong{\sphinxupquote{z\_active}} \textendash{} z\sphinxhyphen{}coordinates (Stationshoehe): \sphinxstylestrong{as array}. Should be set from DwdDataPrep.data\_prep\_for\_new\_location()

\item {} 
\sphinxAtStartPar
\sphinxstyleliteralstrong{\sphinxupquote{zip\_data\_active}} \textendash{} \sphinxstylestrong{as array}. Zipped x\sphinxhyphen{}coordinates and y\sphinxhyphen{}coordinates.

\item {} 
\sphinxAtStartPar
\sphinxstyleliteralstrong{\sphinxupquote{k\_factor}} \textendash{} \sphinxstylestrong{as int}. How many stations you want to find around your coordinates?

\item {} 
\sphinxAtStartPar
\sphinxstyleliteralstrong{\sphinxupquote{start\_date}} \textendash{} \sphinxstylestrong{as int}. \textendash{}\textgreater{} YYYYMMDDHHMM

\item {} 
\sphinxAtStartPar
\sphinxstyleliteralstrong{\sphinxupquote{end\_date}} \textendash{} \sphinxstylestrong{as int}. \textendash{}\textgreater{} YYYYMMDDHHMM

\item {} 
\sphinxAtStartPar
\sphinxstyleliteralstrong{\sphinxupquote{activ\_id}} \textendash{} \sphinxstylestrong{as list}. All the activ stations in timedelta

\item {} 
\sphinxAtStartPar
\sphinxstyleliteralstrong{\sphinxupquote{station\_list}} \textendash{} \sphinxstylestrong{as dict}. All the stations as an object in a dictionary

\end{itemize}

\end{description}\end{quote}

\end{fulllineitems}

\index{average\_for\_coordinate() (DwdNearNeighbor.NearNeighbor method)@\spxentry{average\_for\_coordinate()}\spxextra{DwdNearNeighbor.NearNeighbor method}}

\begin{fulllineitems}
\phantomsection\label{\detokenize{DwdNearNeighbor:DwdNearNeighbor.NearNeighbor.average_for_coordinate}}\pysiglinewithargsret{\sphinxbfcode{\sphinxupquote{average\_for\_coordinate}}}{\emph{\DUrole{n}{data\_looking\_for}\DUrole{o}{=}\DUrole{default_value}{\textquotesingle{}TT\_10\textquotesingle{}}}, \emph{\DUrole{n}{qn\_weight\_n\_avg}\DUrole{o}{=}\DUrole{default_value}{False}}, \emph{\DUrole{n}{distance\_weight\_n\_avg}\DUrole{o}{=}\DUrole{default_value}{False}}, \emph{\DUrole{n}{qn\_distance\_weight\_n\_avg}\DUrole{o}{=}\DUrole{default_value}{False}}, \emph{\DUrole{n}{compare\_n\_avg}\DUrole{o}{=}\DUrole{default_value}{False}}, \emph{\DUrole{n}{compare\_station}\DUrole{o}{=}\DUrole{default_value}{None}}}{}~\begin{quote}\begin{description}
\item[{Description}] \leavevmode
\sphinxAtStartPar
Is calculation different variants.

\item[{Param}] \leavevmode
\sphinxAtStartPar
data\_looking\_for=”TT\_10”: \sphinxstylestrong{as string}. Check Dwd.Dict title\_dict or unit\_dict.

\item[{Param}] \leavevmode
\sphinxAtStartPar
qn\_weight\_n\_avg=False: \sphinxstylestrong{as Boolean}. If True weight the quality of data.

\item[{Param}] \leavevmode
\sphinxAtStartPar
distance\_weight\_n\_avg=False: \sphinxstylestrong{as Boolean}. If True weight the distance of data.

\item[{Param}] \leavevmode
\sphinxAtStartPar
qn\_distance\_weight\_n\_avg=False (not needed for now, maybe in future): \sphinxstylestrong{as Boolean}.

\item[{Param}] \leavevmode
\sphinxAtStartPar
compare\_n\_avg=False: \sphinxstylestrong{as Boolean}. If you want to compare, how accurate your methods are, set True and choose coordinates for a station.

\item[{Param}] \leavevmode
\sphinxAtStartPar
compare\_station=None: \sphinxstylestrong{as string}. If compare\_n\_avg==True set this as the name of a station with the right prefix

\item[{Returns}] \leavevmode
\sphinxAtStartPar
if compare\_n\_avg==True: data\_all (\sphinxstylestrong{as DataFrame}), data\_mean (\sphinxstylestrong{as DataFrame}), index\_for\_plot (\sphinxstylestrong{as list}), column\_name\_list (\sphinxstylestrong{as list}), data\_to\_compare (\sphinxstylestrong{as DataFrame}), diff (\sphinxstylestrong{as DataFrame}), maximum (\sphinxstylestrong{as int}), avg\_diff (\sphinxstylestrong{as array})

\item[{Returns}] \leavevmode
\sphinxAtStartPar
if compare\_n\_avg==False: data\_all (\sphinxstylestrong{as DataFrame}), data\_mean (\sphinxstylestrong{as DataFrame}), index\_for\_plot (\sphinxstylestrong{as list}), column\_name\_list (\sphinxstylestrong{as list})

\end{description}\end{quote}

\end{fulllineitems}

\index{dataframe\_near\_from\_to() (DwdNearNeighbor.NearNeighbor method)@\spxentry{dataframe\_near\_from\_to()}\spxextra{DwdNearNeighbor.NearNeighbor method}}

\begin{fulllineitems}
\phantomsection\label{\detokenize{DwdNearNeighbor:DwdNearNeighbor.NearNeighbor.dataframe_near_from_to}}\pysiglinewithargsret{\sphinxbfcode{\sphinxupquote{dataframe\_near\_from\_to}}}{}{}~\begin{quote}\begin{description}
\item[{Description}] \leavevmode
\sphinxAtStartPar
Is generating a DataFrame with alle the dates inside your files (timedelta for every file) inside your timedelta. Also returning same lists from self.datapath\_near()

\item[{Returns}] \leavevmode
\sphinxAtStartPar
df\_from\_to, datapath\_near\_list, column\_names\_list, column\_name\_list: \sphinxstylestrong{as list, df\_from\_to as DataFrame}

\end{description}\end{quote}

\end{fulllineitems}

\index{dataframe\_near\_from\_to\_path() (DwdNearNeighbor.NearNeighbor method)@\spxentry{dataframe\_near\_from\_to\_path()}\spxextra{DwdNearNeighbor.NearNeighbor method}}

\begin{fulllineitems}
\phantomsection\label{\detokenize{DwdNearNeighbor:DwdNearNeighbor.NearNeighbor.dataframe_near_from_to_path}}\pysiglinewithargsret{\sphinxbfcode{\sphinxupquote{dataframe\_near\_from\_to\_path}}}{}{}~\begin{quote}\begin{description}
\item[{Description}] \leavevmode
\sphinxAtStartPar
Is connecting df\_from\_to with datapah\_near\_list for easier handling.

\item[{Returns}] \leavevmode
\sphinxAtStartPar
df\_from\_to, column\_names\_list, column\_name\_list: \sphinxstylestrong{as list, df\_from\_to as DataFrame}

\end{description}\end{quote}

\end{fulllineitems}

\index{datapath\_near() (DwdNearNeighbor.NearNeighbor method)@\spxentry{datapath\_near()}\spxextra{DwdNearNeighbor.NearNeighbor method}}

\begin{fulllineitems}
\phantomsection\label{\detokenize{DwdNearNeighbor:DwdNearNeighbor.NearNeighbor.datapath_near}}\pysiglinewithargsret{\sphinxbfcode{\sphinxupquote{datapath\_near}}}{}{}~\begin{quote}\begin{description}
\item[{Description}] \leavevmode
\sphinxAtStartPar
Is getting all the paths for data of all the near stations in timedelta. Also creating a list with the columnnames for your DataFrame

\item[{Returns}] \leavevmode
\sphinxAtStartPar
datapath\_near\_list, column\_names\_list, column\_name\_list: \sphinxstylestrong{all as list}

\end{description}\end{quote}

\end{fulllineitems}

\index{date\_range\_df() (DwdNearNeighbor.NearNeighbor method)@\spxentry{date\_range\_df()}\spxextra{DwdNearNeighbor.NearNeighbor method}}

\begin{fulllineitems}
\phantomsection\label{\detokenize{DwdNearNeighbor:DwdNearNeighbor.NearNeighbor.date_range_df}}\pysiglinewithargsret{\sphinxbfcode{\sphinxupquote{date\_range\_df}}}{}{}~\begin{quote}\begin{description}
\item[{Description}] \leavevmode
\sphinxAtStartPar
Is generating an empty DataFrame for timedelta. If you want to change the resolution to 1 minute, you should make some changes here

\item[{Returns}] \leavevmode
\sphinxAtStartPar
date\_range\_df: \sphinxstylestrong{as DataFrame}

\end{description}\end{quote}

\end{fulllineitems}

\index{find\_near() (DwdNearNeighbor.NearNeighbor method)@\spxentry{find\_near()}\spxextra{DwdNearNeighbor.NearNeighbor method}}

\begin{fulllineitems}
\phantomsection\label{\detokenize{DwdNearNeighbor:DwdNearNeighbor.NearNeighbor.find_near}}\pysiglinewithargsret{\sphinxbfcode{\sphinxupquote{find\_near}}}{}{}~\begin{quote}\begin{description}
\item[{Description}] \leavevmode
\sphinxAtStartPar
Will find all the near stations around your location.

\item[{Returns}] \leavevmode
\sphinxAtStartPar
x\_near, y\_near, z\_near, activ\_near\_id: \sphinxstylestrong{as array \sphinxhyphen{} activ\_near\_id as list}

\end{description}\end{quote}

\end{fulllineitems}


\end{fulllineitems}



\section{DwdPlotter module}
\label{\detokenize{DwdPlotter:module-DwdPlotter}}\label{\detokenize{DwdPlotter:dwdplotter-module}}\label{\detokenize{DwdPlotter::doc}}\index{module@\spxentry{module}!DwdPlotter@\spxentry{DwdPlotter}}\index{DwdPlotter@\spxentry{DwdPlotter}!module@\spxentry{module}}\index{PlotterForData (class in DwdPlotter)@\spxentry{PlotterForData}\spxextra{class in DwdPlotter}}

\begin{fulllineitems}
\phantomsection\label{\detokenize{DwdPlotter:DwdPlotter.PlotterForData}}\pysiglinewithargsret{\sphinxbfcode{\sphinxupquote{class }}\sphinxcode{\sphinxupquote{DwdPlotter.}}\sphinxbfcode{\sphinxupquote{PlotterForData}}}{\emph{\DUrole{n}{data\_all}}, \emph{\DUrole{n}{data\_mean}}, \emph{\DUrole{n}{index\_for\_plot}}, \emph{\DUrole{n}{column\_name\_list}}, \emph{\DUrole{n}{start\_date\_datetime}}, \emph{\DUrole{n}{end\_date\_datetime}}, \emph{\DUrole{n}{plot\_name}}, \emph{\DUrole{n}{k\_factor}}, \emph{\DUrole{n}{x\_coordinate}}, \emph{\DUrole{n}{y\_coordinate}}, \emph{\DUrole{n}{type\_of\_data}}, \emph{\DUrole{n}{unit\_dict}}, \emph{\DUrole{n}{title\_dict}}}{}
\sphinxAtStartPar
Bases: \sphinxcode{\sphinxupquote{object}}
\begin{quote}\begin{description}
\item[{Description}] \leavevmode
\sphinxAtStartPar
This class will create plots of the data of chosen stations.

\end{description}\end{quote}
\index{\_\_init\_\_() (DwdPlotter.PlotterForData method)@\spxentry{\_\_init\_\_()}\spxextra{DwdPlotter.PlotterForData method}}

\begin{fulllineitems}
\phantomsection\label{\detokenize{DwdPlotter:DwdPlotter.PlotterForData.__init__}}\pysiglinewithargsret{\sphinxbfcode{\sphinxupquote{\_\_init\_\_}}}{\emph{\DUrole{n}{data\_all}}, \emph{\DUrole{n}{data\_mean}}, \emph{\DUrole{n}{index\_for\_plot}}, \emph{\DUrole{n}{column\_name\_list}}, \emph{\DUrole{n}{start\_date\_datetime}}, \emph{\DUrole{n}{end\_date\_datetime}}, \emph{\DUrole{n}{plot\_name}}, \emph{\DUrole{n}{k\_factor}}, \emph{\DUrole{n}{x\_coordinate}}, \emph{\DUrole{n}{y\_coordinate}}, \emph{\DUrole{n}{type\_of\_data}}, \emph{\DUrole{n}{unit\_dict}}, \emph{\DUrole{n}{title\_dict}}}{}~\begin{quote}\begin{description}
\item[{Parameters}] \leavevmode\begin{itemize}
\item {} 
\sphinxAtStartPar
\sphinxstyleliteralstrong{\sphinxupquote{data\_all}} \textendash{} \sphinxstylestrong{as DataFrame}. All the datas in a DataFrame.

\item {} 
\sphinxAtStartPar
\sphinxstyleliteralstrong{\sphinxupquote{data\_mean}} \textendash{} \sphinxstylestrong{as DataFrame}. Your calculation (average for example).

\item {} 
\sphinxAtStartPar
\sphinxstyleliteralstrong{\sphinxupquote{index\_for\_plot}} \textendash{} \sphinxstylestrong{as list}. The timedelta for x\sphinxhyphen{}axes.

\item {} 
\sphinxAtStartPar
\sphinxstyleliteralstrong{\sphinxupquote{column\_name\_list}} \textendash{} \sphinxstylestrong{as list}. Names of plotted lines.

\item {} 
\sphinxAtStartPar
\sphinxstyleliteralstrong{\sphinxupquote{start\_date\_datetime}} \textendash{} \sphinxstylestrong{as datetime}. Your chosen start\sphinxhyphen{}date (YYYY\sphinxhyphen{}MM\sphinxhyphen{}DD\sphinxhyphen{}HH:MM).

\item {} 
\sphinxAtStartPar
\sphinxstyleliteralstrong{\sphinxupquote{end\_date\_datetime}} \textendash{} \sphinxstylestrong{as datetime}. Your chosen end\sphinxhyphen{}date (YYYY\sphinxhyphen{}MM\sphinxhyphen{}DD\sphinxhyphen{}HH:MM).

\item {} 
\sphinxAtStartPar
\sphinxstyleliteralstrong{\sphinxupquote{plot\_name}} \textendash{} \sphinxstylestrong{as string}.  Your chosen plot\sphinxhyphen{}title.

\item {} 
\sphinxAtStartPar
\sphinxstyleliteralstrong{\sphinxupquote{k\_factor}} \textendash{} \sphinxstylestrong{as int}. Your chosen k\_factor.

\item {} 
\sphinxAtStartPar
\sphinxstyleliteralstrong{\sphinxupquote{x\_coordinate}} \textendash{} \sphinxstylestrong{as float}. Your chosen x\sphinxhyphen{}coordinate.

\item {} 
\sphinxAtStartPar
\sphinxstyleliteralstrong{\sphinxupquote{y\_coordinate}} \textendash{} \sphinxstylestrong{as float}. Your chosen y\sphinxhyphen{}coordinate.

\item {} 
\sphinxAtStartPar
\sphinxstyleliteralstrong{\sphinxupquote{type\_of\_data}} \textendash{} \sphinxstylestrong{as string}. Check DwdDicht.py type\_of\_data\_list

\item {} 
\sphinxAtStartPar
\sphinxstyleliteralstrong{\sphinxupquote{unit\_dict}} \textendash{} \sphinxstylestrong{as string}. Will get the correct type from self.type\_of\_data.

\item {} 
\sphinxAtStartPar
\sphinxstyleliteralstrong{\sphinxupquote{title\_dict}} \textendash{} \sphinxstylestrong{as string}. Will get the correct type from self.type\_of\_data.

\end{itemize}

\end{description}\end{quote}

\end{fulllineitems}

\index{plotting\_compare() (DwdPlotter.PlotterForData method)@\spxentry{plotting\_compare()}\spxextra{DwdPlotter.PlotterForData method}}

\begin{fulllineitems}
\phantomsection\label{\detokenize{DwdPlotter:DwdPlotter.PlotterForData.plotting_compare}}\pysiglinewithargsret{\sphinxbfcode{\sphinxupquote{plotting\_compare}}}{\emph{\DUrole{n}{compare\_station}}, \emph{\DUrole{n}{data\_to\_compare}}, \emph{\DUrole{n}{diff}}, \emph{\DUrole{n}{maximum}}, \emph{\DUrole{n}{avg\_diff}}, \emph{\DUrole{n}{type\_of\_method}}}{}~\begin{quote}\begin{description}
\item[{Description}] \leavevmode
\sphinxAtStartPar
Will plot your data and the data with a station you are comparing with.

\item[{Parameters}] \leavevmode\begin{itemize}
\item {} 
\sphinxAtStartPar
\sphinxstyleliteralstrong{\sphinxupquote{compare\_station}} \textendash{} \sphinxstylestrong{as string}. The name of the station with the right prefix.

\item {} 
\sphinxAtStartPar
\sphinxstyleliteralstrong{\sphinxupquote{data\_to\_compare}} \textendash{} \sphinxstylestrong{as DataFrame}. The real data of a station.

\item {} 
\sphinxAtStartPar
\sphinxstyleliteralstrong{\sphinxupquote{diff}} \textendash{} \sphinxstylestrong{as DataFrame}. The difference between your calculation and real data of a station.

\item {} 
\sphinxAtStartPar
\sphinxstyleliteralstrong{\sphinxupquote{maximum}} \textendash{} \sphinxstylestrong{as int}. Maximum difference.

\item {} 
\sphinxAtStartPar
\sphinxstyleliteralstrong{\sphinxupquote{avg\_diff}} \textendash{} \sphinxstylestrong{as array}. Average difference. (Will plot a constant line)

\item {} 
\sphinxAtStartPar
\sphinxstyleliteralstrong{\sphinxupquote{type\_of\_method}} \textendash{} \sphinxstylestrong{as string}. Name of your calculation method.

\end{itemize}

\item[{Returns}] \leavevmode
\sphinxAtStartPar
“plot saved” if succeeded.

\end{description}\end{quote}

\end{fulllineitems}

\index{plotting\_data() (DwdPlotter.PlotterForData method)@\spxentry{plotting\_data()}\spxextra{DwdPlotter.PlotterForData method}}

\begin{fulllineitems}
\phantomsection\label{\detokenize{DwdPlotter:DwdPlotter.PlotterForData.plotting_data}}\pysiglinewithargsret{\sphinxbfcode{\sphinxupquote{plotting\_data}}}{\emph{\DUrole{n}{type\_of\_method}}}{}~\begin{quote}\begin{description}
\item[{Description}] \leavevmode
\sphinxAtStartPar
Will plot your data.

\item[{Parameters}] \leavevmode
\sphinxAtStartPar
\sphinxstyleliteralstrong{\sphinxupquote{type\_of\_method}} \textendash{} \sphinxstylestrong{as string} Name of your calculation method

\item[{Returns}] \leavevmode
\sphinxAtStartPar
“plot saved” if succeeded.

\end{description}\end{quote}

\end{fulllineitems}


\end{fulllineitems}

\index{PlotterForStations (class in DwdPlotter)@\spxentry{PlotterForStations}\spxextra{class in DwdPlotter}}

\begin{fulllineitems}
\phantomsection\label{\detokenize{DwdPlotter:DwdPlotter.PlotterForStations}}\pysiglinewithargsret{\sphinxbfcode{\sphinxupquote{class }}\sphinxcode{\sphinxupquote{DwdPlotter.}}\sphinxbfcode{\sphinxupquote{PlotterForStations}}}{\emph{\DUrole{n}{x}}, \emph{\DUrole{n}{y}}, \emph{\DUrole{n}{z}\DUrole{o}{=}\DUrole{default_value}{0}}, \emph{\DUrole{n}{type\_of\_data}\DUrole{o}{=}\DUrole{default_value}{\textquotesingle{}\textquotesingle{}}}}{}
\sphinxAtStartPar
Bases: \sphinxcode{\sphinxupquote{object}}
\begin{quote}\begin{description}
\item[{Description}] \leavevmode
\sphinxAtStartPar
This class will create plots of station locations

\end{description}\end{quote}
\index{\_\_init\_\_() (DwdPlotter.PlotterForStations method)@\spxentry{\_\_init\_\_()}\spxextra{DwdPlotter.PlotterForStations method}}

\begin{fulllineitems}
\phantomsection\label{\detokenize{DwdPlotter:DwdPlotter.PlotterForStations.__init__}}\pysiglinewithargsret{\sphinxbfcode{\sphinxupquote{\_\_init\_\_}}}{\emph{\DUrole{n}{x}}, \emph{\DUrole{n}{y}}, \emph{\DUrole{n}{z}\DUrole{o}{=}\DUrole{default_value}{0}}, \emph{\DUrole{n}{type\_of\_data}\DUrole{o}{=}\DUrole{default_value}{\textquotesingle{}\textquotesingle{}}}}{}~\begin{quote}\begin{description}
\item[{Parameters}] \leavevmode\begin{itemize}
\item {} 
\sphinxAtStartPar
\sphinxstyleliteralstrong{\sphinxupquote{x}} \textendash{} x\sphinxhyphen{}coordinates: \sphinxstylestrong{as array}.

\item {} 
\sphinxAtStartPar
\sphinxstyleliteralstrong{\sphinxupquote{y}} \textendash{} y\sphinxhyphen{}coordinates: \sphinxstylestrong{as array}.

\item {} 
\sphinxAtStartPar
\sphinxstyleliteralstrong{\sphinxupquote{z}} \textendash{} z\sphinxhyphen{}coordinates: \sphinxstylestrong{as array}.

\item {} 
\sphinxAtStartPar
\sphinxstyleliteralstrong{\sphinxupquote{type\_of\_data}} \textendash{} \sphinxstylestrong{as string}. Check DwdDict.py type\_of\_data\_list

\end{itemize}

\end{description}\end{quote}

\end{fulllineitems}

\index{plotting\_3d() (DwdPlotter.PlotterForStations method)@\spxentry{plotting\_3d()}\spxextra{DwdPlotter.PlotterForStations method}}

\begin{fulllineitems}
\phantomsection\label{\detokenize{DwdPlotter:DwdPlotter.PlotterForStations.plotting_3d}}\pysiglinewithargsret{\sphinxbfcode{\sphinxupquote{plotting\_3d}}}{\emph{\DUrole{n}{projection}\DUrole{o}{=}\DUrole{default_value}{False}}}{}~\begin{quote}\begin{description}
\item[{Description}] \leavevmode
\sphinxAtStartPar
This method will plot all the stations in as a 3D\sphinxhyphen{}Version

\item[{Parameters}] \leavevmode
\sphinxAtStartPar
\sphinxstyleliteralstrong{\sphinxupquote{projection}} \textendash{} \sphinxstylestrong{as Boolean}. If projection==True, it will project the height of the stations.

\item[{Returns}] \leavevmode
\sphinxAtStartPar
“plot saved” if succeeded.

\end{description}\end{quote}

\end{fulllineitems}

\index{plotting\_height\_2d() (DwdPlotter.PlotterForStations method)@\spxentry{plotting\_height\_2d()}\spxextra{DwdPlotter.PlotterForStations method}}

\begin{fulllineitems}
\phantomsection\label{\detokenize{DwdPlotter:DwdPlotter.PlotterForStations.plotting_height_2d}}\pysiglinewithargsret{\sphinxbfcode{\sphinxupquote{plotting\_height\_2d}}}{}{}~\begin{quote}\begin{description}
\item[{Description}] \leavevmode
\sphinxAtStartPar
This method will plot all the stations in as a 2D\sphinxhyphen{}Version

\item[{Returns}] \leavevmode
\sphinxAtStartPar
“plot saved” if succeeded.

\end{description}\end{quote}

\end{fulllineitems}


\end{fulllineitems}



\section{main module}
\label{\detokenize{main:module-main}}\label{\detokenize{main:main-module}}\label{\detokenize{main::doc}}\index{module@\spxentry{module}!main@\spxentry{main}}\index{main@\spxentry{main}!module@\spxentry{module}}
\begin{sphinxShadowBox}
\sphinxstyletopictitle{Table of Contents (for quick start)}
\begin{itemize}
\item {} 
\sphinxAtStartPar
\phantomsection\label{\detokenize{quick_start:id1}}{\hyperref[\detokenize{quick_start:quick-start}]{\sphinxcrossref{Quick\sphinxhyphen{}Start}}}
\begin{itemize}
\item {} 
\sphinxAtStartPar
\phantomsection\label{\detokenize{quick_start:id2}}{\hyperref[\detokenize{quick_start:getting-started-with-dwdmain-module}]{\sphinxcrossref{getting started with DwdMain module}}}
\begin{itemize}
\item {} 
\sphinxAtStartPar
\phantomsection\label{\detokenize{quick_start:id3}}{\hyperref[\detokenize{quick_start:getting-started-download-air-temperature}]{\sphinxcrossref{getting started: download air\_temperature}}}

\item {} 
\sphinxAtStartPar
\phantomsection\label{\detokenize{quick_start:id4}}{\hyperref[\detokenize{quick_start:getting-started-download-all-air-temperatur-solar-wind-precipitation}]{\sphinxcrossref{getting started: download all (air\_temperatur, solar, wind, precipitation)}}}

\item {} 
\sphinxAtStartPar
\phantomsection\label{\detokenize{quick_start:id5}}{\hyperref[\detokenize{quick_start:getting-started-some-station-information}]{\sphinxcrossref{getting started: some station information}}}

\item {} 
\sphinxAtStartPar
\phantomsection\label{\detokenize{quick_start:id6}}{\hyperref[\detokenize{quick_start:getting-started-plots}]{\sphinxcrossref{getting started: plots}}}

\item {} 
\sphinxAtStartPar
\phantomsection\label{\detokenize{quick_start:id7}}{\hyperref[\detokenize{quick_start:getting-started-maps}]{\sphinxcrossref{getting started: maps}}}

\item {} 
\sphinxAtStartPar
\phantomsection\label{\detokenize{quick_start:id8}}{\hyperref[\detokenize{quick_start:getting-started-data-plots-and-calculations}]{\sphinxcrossref{getting started: data\sphinxhyphen{}plots and calculations}}}

\end{itemize}

\end{itemize}

\item {} 
\sphinxAtStartPar
\phantomsection\label{\detokenize{quick_start:id9}}{\hyperref[\detokenize{quick_start:indices-and-tables}]{\sphinxcrossref{Indices and tables}}}

\end{itemize}
\end{sphinxShadowBox}


\chapter{Quick\sphinxhyphen{}Start}
\label{\detokenize{quick_start:quick-start}}\label{\detokenize{quick_start::doc}}

\section{getting started with DwdMain module}
\label{\detokenize{quick_start:getting-started-with-dwdmain-module}}

\subsection{getting started: download air\_temperature}
\label{\detokenize{quick_start:getting-started-download-air-temperature}}
\begin{sphinxadmonition}{warning}{Warning:}
\begin{DUlineblock}{0em}
\item[] \sphinxstylestrong{For windows user}: To avoid any issues with the path limit of windows, you should enable unlimited path on your machine.
\item[] Press Win+R \sphinxhyphen{}\textgreater{} REGEDIT \sphinxhyphen{}\textgreater{} HKEY\_LOCAL\_MACHINE\textbackslash{}SYSTEM\textbackslash{}CurrentControlSet\textbackslash{}Control\textbackslash{}FileSystem \sphinxhyphen{}\textgreater{} LongPathEnabled \sphinxhyphen{}\textgreater{} set “1”.
\item[] If LongPathEnabled doesn’t show in your FileSystem, you can create a new REG\_DWORD (DWORD (32\sphinxhyphen{}Bit)).
\end{DUlineblock}
\end{sphinxadmonition}

\noindent\sphinxincludegraphics[width=1000\sphinxpxdimen]{{disable_windwos_path_limit}.png}

\begin{sphinxVerbatim}[commandchars=\\\{\}]
\PYG{k+kn}{from} \PYG{n+nn}{DwdMain} \PYG{k+kn}{import} \PYG{n}{main\PYGZus{}dwd}
\PYG{n}{main\PYGZus{}dwd}\PYG{p}{(}\PYG{n}{local\PYGZus{}domain}\PYG{o}{=}\PYG{l+s+s2}{\PYGZdq{}}\PYG{l+s+s2}{YOUR\PYGZus{}PATH/}\PYG{l+s+s2}{\PYGZdq{}}\PYG{p}{,} \PYG{n}{type\PYGZus{}of\PYGZus{}data}\PYG{o}{=}\PYG{l+s+s2}{\PYGZdq{}}\PYG{l+s+s2}{air\PYGZus{}temperature}\PYG{l+s+s2}{\PYGZdq{}}\PYG{p}{)}\PYG{o}{.}\PYG{n}{main\PYGZus{}datascrapper}\PYG{p}{(}\PYG{n+nb}{all}\PYG{o}{=}\PYG{k+kc}{False}\PYG{p}{)}
\PYG{c+c1}{\PYGZsh{} Will download and unzip the data for air\PYGZus{}temperature (historical, meta\PYGZus{}data, now and recent) to YOUR\PYGZus{}PATH/.}
\PYG{n}{main\PYGZus{}dwd}\PYG{p}{(}\PYG{n}{local\PYGZus{}domain}\PYG{o}{=}\PYG{l+s+s2}{\PYGZdq{}}\PYG{l+s+s2}{YOUR\PYGZus{}PATH/}\PYG{l+s+s2}{\PYGZdq{}}\PYG{p}{,} \PYG{n}{type\PYGZus{}of\PYGZus{}data}\PYG{o}{=}\PYG{l+s+s2}{\PYGZdq{}}\PYG{l+s+s2}{air\PYGZus{}temperature}\PYG{l+s+s2}{\PYGZdq{}}\PYG{p}{)}\PYG{o}{.}\PYG{n}{main\PYGZus{}writer}\PYG{p}{(}\PYG{n}{type\PYGZus{}of\PYGZus{}data\PYGZus{}list}\PYG{o}{=}\PYG{p}{[}\PYG{l+s+s2}{\PYGZdq{}}\PYG{l+s+s2}{air\PYGZus{}temperature}\PYG{l+s+s2}{\PYGZdq{}}\PYG{p}{]}\PYG{p}{,} \PYG{n+nb}{all}\PYG{o}{=}\PYG{k+kc}{False}\PYG{p}{)}
\PYG{c+c1}{\PYGZsh{} Will create some .json files inside .../extracted\PYGZus{}files/ for faster loading times. This step is important. This .json files includes dates and paths for every station.}
\end{sphinxVerbatim}


\subsection{getting started: download all (air\_temperatur, solar, wind, precipitation)}
\label{\detokenize{quick_start:getting-started-download-all-air-temperatur-solar-wind-precipitation}}
\begin{sphinxVerbatim}[commandchars=\\\{\}]
\PYG{k+kn}{from} \PYG{n+nn}{DwdMain} \PYG{k+kn}{import} \PYG{n}{main\PYGZus{}dwd}
\PYG{n}{main\PYGZus{}dwd}\PYG{p}{(}\PYG{n}{local\PYGZus{}domain}\PYG{o}{=}\PYG{l+s+s2}{\PYGZdq{}}\PYG{l+s+s2}{YOUR\PYGZus{}PATH/}\PYG{l+s+s2}{\PYGZdq{}}\PYG{p}{)}\PYG{o}{.}\PYG{n}{main\PYGZus{}datascrapper}\PYG{p}{(}\PYG{n+nb}{all}\PYG{o}{=}\PYG{k+kc}{True}\PYG{p}{)}
\PYG{c+c1}{\PYGZsh{} Will download and unzip all the data for air\PYGZus{}temperatur, solar, wind, precipitation (historical, meta\PYGZus{}data, now and recent) to YOUR\PYGZus{}PATH/.}
\PYG{n}{main\PYGZus{}dwd}\PYG{p}{(}\PYG{n}{local\PYGZus{}domain}\PYG{o}{=}\PYG{l+s+s2}{\PYGZdq{}}\PYG{l+s+s2}{YOUR\PYGZus{}PATH/}\PYG{l+s+s2}{\PYGZdq{}}\PYG{p}{)}\PYG{o}{.}\PYG{n}{main\PYGZus{}writer}\PYG{p}{(}\PYG{n+nb}{all}\PYG{o}{=}\PYG{k+kc}{True}\PYG{p}{)}
\PYG{c+c1}{\PYGZsh{} Will create some .json files inside .../extracted\PYGZus{}files/ for faster loading times. This step is important. Inside this .json files will dates and paths for every station.}
\end{sphinxVerbatim}


\subsection{getting started: some station information}
\label{\detokenize{quick_start:getting-started-some-station-information}}
\begin{sphinxVerbatim}[commandchars=\\\{\}]
\PYG{k+kn}{from} \PYG{n+nn}{DwdMain} \PYG{k+kn}{import} \PYG{n}{main\PYGZus{}dwd}
\PYG{n+nb}{print}\PYG{p}{(}\PYG{n}{main\PYGZus{}dwd}\PYG{p}{(}\PYG{n}{local\PYGZus{}domain}\PYG{o}{=}\PYG{l+s+s2}{\PYGZdq{}}\PYG{l+s+s2}{YOUR\PYGZus{}PATH/}\PYG{l+s+s2}{\PYGZdq{}}\PYG{p}{,}\PYG{n}{type\PYGZus{}of\PYGZus{}data}\PYG{o}{=}\PYG{l+s+s2}{\PYGZdq{}}\PYG{l+s+s2}{air\PYGZus{}temperature}\PYG{l+s+s2}{\PYGZdq{}}\PYG{p}{,} \PYG{n}{type\PYGZus{}of\PYGZus{}time}\PYG{o}{=}\PYG{l+s+s2}{\PYGZdq{}}\PYG{l+s+s2}{historical}\PYG{l+s+s2}{\PYGZdq{}}\PYG{p}{)}\PYG{o}{.}\PYG{n}{main\PYGZus{}station\PYGZus{}information}\PYG{p}{(}\PYG{l+s+s2}{\PYGZdq{}}\PYG{l+s+s2}{TU\PYGZus{}00003}\PYG{l+s+s2}{\PYGZdq{}}\PYG{p}{)}\PYG{p}{)}
\PYG{c+c1}{\PYGZsh{} Let\PYGZsq{}s get some information about a station.}

\PYG{k}{return}\PYG{p}{:}
\PYG{p}{\PYGZob{}}\PYG{l+s+s1}{\PYGZsq{}}\PYG{l+s+s1}{ID}\PYG{l+s+s1}{\PYGZsq{}}\PYG{p}{:} \PYG{l+s+s1}{\PYGZsq{}}\PYG{l+s+s1}{TU\PYGZus{}00003}\PYG{l+s+s1}{\PYGZsq{}}\PYG{p}{,} \PYG{l+s+s1}{\PYGZsq{}}\PYG{l+s+s1}{von\PYGZus{}datum}\PYG{l+s+s1}{\PYGZsq{}}\PYG{p}{:} \PYG{l+s+s1}{\PYGZsq{}}\PYG{l+s+s1}{1993\PYGZhy{}4\PYGZhy{}29}\PYG{l+s+s1}{\PYGZsq{}}\PYG{p}{,} \PYG{l+s+s1}{\PYGZsq{}}\PYG{l+s+s1}{bis\PYGZus{}datum}\PYG{l+s+s1}{\PYGZsq{}}\PYG{p}{:} \PYG{l+s+s1}{\PYGZsq{}}\PYG{l+s+s1}{2011\PYGZhy{}3\PYGZhy{}31}\PYG{l+s+s1}{\PYGZsq{}}\PYG{p}{,} \PYG{l+s+s1}{\PYGZsq{}}\PYG{l+s+s1}{stationshoehe}\PYG{l+s+s1}{\PYGZsq{}}\PYG{p}{:} \PYG{l+m+mi}{202}\PYG{p}{,}
\PYG{l+s+s1}{\PYGZsq{}}\PYG{l+s+s1}{geoBreite}\PYG{l+s+s1}{\PYGZsq{}}\PYG{p}{:} \PYG{l+m+mf}{50.7827}\PYG{p}{,} \PYG{l+s+s1}{\PYGZsq{}}\PYG{l+s+s1}{geoLaenge}\PYG{l+s+s1}{\PYGZsq{}}\PYG{p}{:} \PYG{l+m+mf}{6.0941}\PYG{p}{,} \PYG{l+s+s1}{\PYGZsq{}}\PYG{l+s+s1}{Stationsname}\PYG{l+s+s1}{\PYGZsq{}}\PYG{p}{:} \PYG{l+s+sa}{b}\PYG{l+s+s1}{\PYGZsq{}}\PYG{l+s+s1}{Aachen}\PYG{l+s+s1}{\PYGZsq{}}\PYG{p}{,} \PYG{l+s+s1}{\PYGZsq{}}\PYG{l+s+s1}{Bundesland}\PYG{l+s+s1}{\PYGZsq{}}\PYG{p}{:} \PYG{l+s+sa}{b}\PYG{l+s+s1}{\PYGZsq{}}\PYG{l+s+s1}{Nordrhein\PYGZhy{}Westfalen}\PYG{l+s+s1}{\PYGZsq{}}\PYG{p}{,} \PYG{l+s+s1}{\PYGZsq{}}\PYG{l+s+s1}{Aktivität}\PYG{l+s+s1}{\PYGZsq{}}\PYG{p}{:} \PYG{k+kc}{False}\PYG{p}{\PYGZcb{}}
\PYG{c+c1}{\PYGZsh{} Interesting... what else can we do?}


\PYG{n+nb}{print}\PYG{p}{(}\PYG{n}{main\PYGZus{}dwd}\PYG{p}{(}\PYG{n}{local\PYGZus{}domain}\PYG{o}{=}\PYG{l+s+s2}{\PYGZdq{}}\PYG{l+s+s2}{YOUR\PYGZus{}PATH/}\PYG{l+s+s2}{\PYGZdq{}}\PYG{p}{,}\PYG{n}{type\PYGZus{}of\PYGZus{}data}\PYG{o}{=}\PYG{l+s+s2}{\PYGZdq{}}\PYG{l+s+s2}{air\PYGZus{}temperature}\PYG{l+s+s2}{\PYGZdq{}}\PYG{p}{,} \PYG{n}{type\PYGZus{}of\PYGZus{}time}\PYG{o}{=}\PYG{l+s+s2}{\PYGZdq{}}\PYG{l+s+s2}{historical}\PYG{l+s+s2}{\PYGZdq{}}\PYG{p}{)}\PYG{o}{.}\PYG{n}{main\PYGZus{}station\PYGZus{}array}\PYG{p}{(}\PYG{p}{)}\PYG{p}{)}
\PYG{c+c1}{\PYGZsh{} Will return an array with all available stations for this type of data and this type of time.}

\PYG{k}{return}\PYG{p}{:}
\PYG{p}{[}\PYG{l+s+s1}{\PYGZsq{}}\PYG{l+s+s1}{TU\PYGZus{}00003}\PYG{l+s+s1}{\PYGZsq{}} \PYG{l+s+s1}{\PYGZsq{}}\PYG{l+s+s1}{TU\PYGZus{}00044}\PYG{l+s+s1}{\PYGZsq{}} \PYG{l+s+s1}{\PYGZsq{}}\PYG{l+s+s1}{TU\PYGZus{}00071}\PYG{l+s+s1}{\PYGZsq{}} \PYG{l+s+s1}{\PYGZsq{}}\PYG{l+s+s1}{TU\PYGZus{}00073}\PYG{l+s+s1}{\PYGZsq{}} \PYG{l+s+s1}{\PYGZsq{}}\PYG{l+s+s1}{TU\PYGZus{}00078}\PYG{l+s+s1}{\PYGZsq{}} \PYG{l+s+s1}{\PYGZsq{}}\PYG{l+s+s1}{TU\PYGZus{}00091}\PYG{l+s+s1}{\PYGZsq{}} \PYG{o}{.}\PYG{o}{.}\PYG{o}{.} \PYG{l+s+s1}{\PYGZsq{}}\PYG{l+s+s1}{TU\PYGZus{}15555}\PYG{l+s+s1}{\PYGZsq{}} \PYG{l+s+s1}{\PYGZsq{}}\PYG{l+s+s1}{TU\PYGZus{}15813}\PYG{l+s+s1}{\PYGZsq{}} \PYG{l+s+s1}{\PYGZsq{}}\PYG{l+s+s1}{TU\PYGZus{}19171}\PYG{l+s+s1}{\PYGZsq{}} \PYG{l+s+s1}{\PYGZsq{}}\PYG{l+s+s1}{TU\PYGZus{}19172}\PYG{l+s+s1}{\PYGZsq{}}\PYG{p}{]}
\PYG{c+c1}{\PYGZsh{} What about active stations in my timedelta?}


\PYG{n}{start\PYGZus{}data\PYGZus{}} \PYG{o}{=} \PYG{l+m+mi}{199401190000}
\PYG{n}{end\PYGZus{}date} \PYG{o}{=} \PYG{l+m+mi}{199501010000}
\PYG{n+nb}{print}\PYG{p}{(}\PYG{n}{main\PYGZus{}dwd}\PYG{p}{(}\PYG{n}{local\PYGZus{}domain}\PYG{o}{=}\PYG{l+s+s2}{\PYGZdq{}}\PYG{l+s+s2}{YOUR\PYGZus{}PATH/}\PYG{l+s+s2}{\PYGZdq{}}\PYG{p}{,}
               \PYG{n}{type\PYGZus{}of\PYGZus{}data}\PYG{o}{=}\PYG{l+s+s2}{\PYGZdq{}}\PYG{l+s+s2}{air\PYGZus{}temperature}\PYG{l+s+s2}{\PYGZdq{}}\PYG{p}{,}
               \PYG{n}{type\PYGZus{}of\PYGZus{}time}\PYG{o}{=}\PYG{l+s+s2}{\PYGZdq{}}\PYG{l+s+s2}{historical}\PYG{l+s+s2}{\PYGZdq{}}\PYG{p}{,}
               \PYG{n}{start\PYGZus{}date}\PYG{o}{=}\PYG{n}{start\PYGZus{}date\PYGZus{}}\PYG{p}{,} \PYG{n}{end\PYGZus{}date}\PYG{o}{=}\PYG{n}{end\PYGZus{}date\PYGZus{}}\PYG{p}{)}\PYG{o}{.}\PYG{n}{main\PYGZus{}activ\PYGZus{}stations\PYGZus{}in\PYGZus{}date}\PYG{p}{(}\PYG{p}{)}\PYG{p}{)}
\PYG{c+c1}{\PYGZsh{} Will return 3 arrays. Array 1: x\PYGZhy{}coordinates for active stations,}
\PYG{c+c1}{\PYGZsh{} array 2: y\PYGZhy{}coordinates for active stations,}
\PYG{c+c1}{\PYGZsh{} array 3: z\PYGZhy{}coordinates for active stations}
\PYG{c+c1}{\PYGZsh{} array 4: station id\PYGZsq{}s (with prefix) for active stations.}

\PYG{k}{return}\PYG{p}{:}
\PYG{n}{array1}\PYG{p}{:} \PYG{p}{[} \PYG{l+m+mf}{6.0941}\PYG{p}{,} \PYG{l+m+mf}{13.9908}\PYG{p}{,} \PYG{l+m+mf}{13.4344}\PYG{p}{,}\PYG{o}{.}\PYG{o}{.}\PYG{o}{.}\PYG{p}{]}
\PYG{n}{array2}\PYG{p}{:} \PYG{p}{[}\PYG{l+m+mf}{50.7827}\PYG{p}{,} \PYG{l+m+mf}{53.0316}\PYG{p}{,} \PYG{l+m+mf}{54.6791}\PYG{p}{,} \PYG{l+m+mf}{51.3744}\PYG{p}{,}\PYG{o}{.}\PYG{o}{.}\PYG{o}{.}\PYG{p}{]}
\PYG{n}{array3}\PYG{p}{:} \PYG{p}{[} \PYG{l+m+mi}{202}\PYG{p}{,}   \PYG{l+m+mi}{54}\PYG{p}{,}   \PYG{l+m+mi}{42}\PYG{p}{,}  \PYG{l+m+mi}{164}\PYG{p}{,}  \PYG{l+m+mi}{393}\PYG{p}{,}    \PYG{l+m+mi}{3}\PYG{p}{,}\PYG{o}{.}\PYG{o}{.}\PYG{o}{.}\PYG{p}{]}
\PYG{n}{array4}\PYG{p}{:} \PYG{p}{[}\PYG{l+s+s1}{\PYGZsq{}}\PYG{l+s+s1}{TU\PYGZus{}00003}\PYG{l+s+s1}{\PYGZsq{}}\PYG{p}{,} \PYG{l+s+s1}{\PYGZsq{}}\PYG{l+s+s1}{TU\PYGZus{}00164}\PYG{l+s+s1}{\PYGZsq{}}\PYG{p}{,} \PYG{l+s+s1}{\PYGZsq{}}\PYG{l+s+s1}{TU\PYGZus{}00183}\PYG{l+s+s1}{\PYGZsq{}}\PYG{p}{,} \PYG{l+s+s1}{\PYGZsq{}}\PYG{l+s+s1}{TU\PYGZus{}00198}\PYG{l+s+s1}{\PYGZsq{}}\PYG{p}{,}\PYG{o}{.}\PYG{o}{.}\PYG{o}{.}\PYG{p}{]}
\end{sphinxVerbatim}


\subsection{getting started: plots}
\label{\detokenize{quick_start:getting-started-plots}}
\begin{sphinxVerbatim}[commandchars=\\\{\}]
\PYG{k+kn}{from} \PYG{n+nn}{DwdMain} \PYG{k+kn}{import} \PYG{n}{main\PYGZus{}dwd}
\PYG{n}{main\PYGZus{}dwd}\PYG{p}{(}\PYG{n}{local\PYGZus{}domain}\PYG{o}{=}\PYG{l+s+s2}{\PYGZdq{}}\PYG{l+s+s2}{YOUR\PYGZus{}PATH/}\PYG{l+s+s2}{\PYGZdq{}}\PYG{p}{,} \PYG{n}{type\PYGZus{}of\PYGZus{}data}\PYG{o}{=}\PYG{l+s+s2}{\PYGZdq{}}\PYG{l+s+s2}{air\PYGZus{}temperature}\PYG{l+s+s2}{\PYGZdq{}}\PYG{p}{,} \PYG{n}{type\PYGZus{}of\PYGZus{}time}\PYG{o}{=}\PYG{l+s+s2}{\PYGZdq{}}\PYG{l+s+s2}{historical}\PYG{l+s+s2}{\PYGZdq{}}\PYG{p}{,}\PYG{p}{)}\PYG{o}{.}\PYG{n}{main\PYGZus{}plotter\PYGZus{}stations}\PYG{p}{(}\PYG{n}{projection}\PYG{o}{=}\PYG{k+kc}{False}\PYG{p}{)}
\PYG{c+c1}{\PYGZsh{} Will plot two graphs. 3D and 2D as a \PYGZdq{}heatmap\PYGZdq{} for all the station of this type of data and this type of time.}
\end{sphinxVerbatim}

\noindent\sphinxincludegraphics[width=500\sphinxpxdimen]{{german_stations_3dair_temperature}.png}

\noindent\sphinxincludegraphics[width=500\sphinxpxdimen]{{german_stations_2d_air_temperature}.png}

\begin{sphinxadmonition}{note}{Note:}
\sphinxAtStartPar
If projection == True, it will project heights in created 3D\sphinxhyphen{}plot on the “Stationshoehe” \sphinxhyphen{} axe.
\end{sphinxadmonition}


\subsection{getting started: maps}
\label{\detokenize{quick_start:getting-started-maps}}
\begin{sphinxVerbatim}[commandchars=\\\{\}]
\PYG{k+kn}{from} \PYG{n+nn}{DwdMain} \PYG{k+kn}{import} \PYG{n}{main\PYGZus{}dwd}
\PYG{n}{local\PYGZus{}domain\PYGZus{}} \PYG{o}{=} \PYG{l+s+sa}{r}\PYG{l+s+s2}{\PYGZdq{}}\PYG{l+s+s2}{YOUR\PYGZus{}PATH/}\PYG{l+s+s2}{\PYGZdq{}}
\PYG{n}{looking\PYGZus{}for\PYGZus{}} \PYG{o}{=} \PYG{p}{[}\PYG{l+s+s2}{\PYGZdq{}}\PYG{l+s+s2}{PP\PYGZus{}10}\PYG{l+s+s2}{\PYGZdq{}}\PYG{p}{]}
\PYG{n}{start\PYGZus{}date\PYGZus{}} \PYG{o}{=} \PYG{l+m+mi}{199401190000}
\PYG{n}{end\PYGZus{}date\PYGZus{}}   \PYG{o}{=} \PYG{l+m+mi}{199501010000}
\PYG{n}{x\PYGZus{}coordinate\PYGZus{}} \PYG{o}{=} \PYG{l+m+mf}{6.0941} \PYG{c+c1}{\PYGZsh{}x\PYGZhy{}coordinates for your location}
\PYG{n}{y\PYGZus{}coordinate\PYGZus{}} \PYG{o}{=} \PYG{l+m+mf}{50.7827} \PYG{c+c1}{\PYGZsh{}y\PYGZhy{}coordinates for your location}
\PYG{n}{z\PYGZus{}coordinate\PYGZus{}} \PYG{o}{=} \PYG{l+m+mi}{0} \PYG{c+c1}{\PYGZsh{} not needed for now (maybe in future)}
\PYG{n}{k\PYGZus{}factor\PYGZus{}} \PYG{o}{=} \PYG{l+m+mi}{10} \PYG{c+c1}{\PYGZsh{} how many station are you looking for around your location? 10 means, it will find 10 next stations for your location}
\PYG{n}{type\PYGZus{}of\PYGZus{}data\PYGZus{}} \PYG{o}{=} \PYG{l+s+s2}{\PYGZdq{}}\PYG{l+s+s2}{air\PYGZus{}temperature}\PYG{l+s+s2}{\PYGZdq{}}
\PYG{n}{type\PYGZus{}of\PYGZus{}time\PYGZus{}} \PYG{o}{=} \PYG{l+s+s2}{\PYGZdq{}}\PYG{l+s+s2}{historical}\PYG{l+s+s2}{\PYGZdq{}}
\PYG{n}{dwd} \PYG{o}{=} \PYG{n}{main\PYGZus{}dwd}\PYG{p}{(} \PYG{n}{local\PYGZus{}domain}\PYG{o}{=}\PYG{n}{local\PYGZus{}domain\PYGZus{}}\PYG{p}{,}
                \PYG{n}{type\PYGZus{}of\PYGZus{}data}\PYG{o}{=}\PYG{n}{type\PYGZus{}of\PYGZus{}data\PYGZus{}}\PYG{p}{,}
                \PYG{n}{type\PYGZus{}of\PYGZus{}time}\PYG{o}{=}\PYG{n}{type\PYGZus{}of\PYGZus{}time\PYGZus{}}\PYG{p}{,}
                \PYG{n}{start\PYGZus{}date}\PYG{o}{=}\PYG{n}{start\PYGZus{}date\PYGZus{}}\PYG{p}{,}
                \PYG{n}{end\PYGZus{}date}\PYG{o}{=}\PYG{n}{end\PYGZus{}date\PYGZus{}}\PYG{p}{,}
                \PYG{n}{x\PYGZus{}coordinate}\PYG{o}{=}\PYG{n}{x\PYGZus{}coordinate\PYGZus{}}\PYG{p}{,}
                \PYG{n}{y\PYGZus{}coordinate}\PYG{o}{=}\PYG{n}{y\PYGZus{}coordinate\PYGZus{}}\PYG{p}{,}
                \PYG{n}{z\PYGZus{}coordinate}\PYG{o}{=}\PYG{n}{z\PYGZus{}coordinate\PYGZus{}}\PYG{p}{,}
                \PYG{n}{k\PYGZus{}factor}\PYG{o}{=}\PYG{n}{k\PYGZus{}factor\PYGZus{}}\PYG{p}{)}
\PYG{n}{dwd}\PYG{o}{.}\PYG{n}{main\PYGZus{}data\PYGZus{}map}\PYG{p}{(}\PYG{p}{)}
\PYG{c+c1}{\PYGZsh{} Is creating some .json files inside the .../MapCreater/}
\PYG{c+c1}{\PYGZsh{} zip\PYGZus{}data\PYGZus{}activ: All the activ stations. (From \PYGZdq{}information\PYGZdq{}.txt)}
\PYG{c+c1}{\PYGZsh{} zip\PYGZus{}data\PYGZus{}active\PYGZus{}in\PYGZus{}date: All the activ stations in your date.}
\PYG{c+c1}{\PYGZsh{} zip\PYGZus{}data\PYGZus{}near: Alle the activ stations around your location.}
\PYG{c+c1}{\PYGZsh{} zip\PYGZus{}data\PYGZus{}not\PYGZus{}activ: All the not activ stations. (From \PYGZdq{}information\PYGZdq{}.txt)}
\PYG{c+c1}{\PYGZsh{} zip\PYGZus{}no\PYGZus{}data: Alle the stations without data. (From \PYGZdq{}information\PYGZdq{}.txt)}
\PYG{c+c1}{\PYGZsh{} Is important for DwdMapCreator}

\PYG{c+c1}{\PYGZsh{} now we saved the data. Let\PYGZsq{}s show it on a map.}
\PYG{k+kn}{from} \PYG{n+nn}{DwdMap} \PYG{k+kn}{import} \PYG{n}{DwdMap}
\PYG{n}{DwdMap}\PYG{p}{(}\PYG{l+s+s2}{\PYGZdq{}}\PYG{l+s+s2}{NearStations}\PYG{l+s+s2}{\PYGZdq{}}\PYG{p}{)}\PYG{o}{.}\PYG{n}{create\PYGZus{}map}\PYG{p}{(}\PYG{p}{)}
\PYG{n}{DwdMap}\PYG{p}{(}\PYG{l+s+s2}{\PYGZdq{}}\PYG{l+s+s2}{Stations}\PYG{l+s+s2}{\PYGZdq{}}\PYG{p}{)}\PYG{o}{.}\PYG{n}{create\PYGZus{}map}\PYG{p}{(}\PYG{p}{)}
\PYG{n}{DwdMap}\PYG{p}{(}\PYG{l+s+s2}{\PYGZdq{}}\PYG{l+s+s2}{ActivInDate}\PYG{l+s+s2}{\PYGZdq{}}\PYG{p}{)}\PYG{o}{.}\PYG{n}{create\PYGZus{}map}\PYG{p}{(}\PYG{p}{)}
\PYG{c+c1}{\PYGZsh{} choose between \PYGZdq{}NearStations\PYGZdq{} , \PYGZdq{}Stations\PYGZdq{} , \PYGZdq{}ActivInDate\PYGZdq{}}
\PYG{c+c1}{\PYGZsh{} \PYGZdq{}NearStations\PYGZdq{}: Will plot all the stations (k\PYGZus{}factor) near your location.}
\PYG{c+c1}{\PYGZsh{} \PYGZdq{}Stations\PYGZdq{}: Will plot all the available stations (for all times)}
\PYG{c+c1}{\PYGZsh{} \PYGZdq{}ActivInDate\PYGZdq{}: Will plot all the activ stations for your timedelta (end\PYGZus{}time \PYGZhy{} start\PYGZus{}time)}
\end{sphinxVerbatim}

\noindent\sphinxincludegraphics[width=500\sphinxpxdimen]{{near_stations}.png}

\noindent\sphinxincludegraphics[width=500\sphinxpxdimen]{{stations}.png}

\noindent\sphinxincludegraphics[width=500\sphinxpxdimen]{{activ_in_date}.png}

\begin{sphinxadmonition}{note}{Note:}
\sphinxAtStartPar
You should open DwdMapCreator with Jupyter Notebooks. Otherwise it won’t show the locations on a map. A lot of red locations on the second map. It means
the data probably not up to date. You should update your data.
\end{sphinxadmonition}


\subsection{getting started: data\sphinxhyphen{}plots and calculations}
\label{\detokenize{quick_start:getting-started-data-plots-and-calculations}}
\begin{sphinxVerbatim}[commandchars=\\\{\}]
\PYG{k+kn}{import} \PYG{n+nn}{os}
\PYG{k+kn}{from} \PYG{n+nn}{DwdMain} \PYG{k+kn}{import} \PYG{n}{main\PYGZus{}dwd}
\PYG{n}{local\PYGZus{}domain\PYGZus{}} \PYG{o}{=} \PYG{l+s+sa}{r}\PYG{l+s+s2}{\PYGZdq{}}\PYG{l+s+s2}{YOUR\PYGZus{}PATH/}\PYG{l+s+s2}{\PYGZdq{}}
\PYG{n}{os}\PYG{o}{.}\PYG{n}{chdir}\PYG{p}{(}\PYG{n}{local\PYGZus{}domain\PYGZus{}}\PYG{p}{)}
\PYG{n}{looking\PYGZus{}for\PYGZus{}} \PYG{o}{=} \PYG{p}{[}\PYG{l+s+s2}{\PYGZdq{}}\PYG{l+s+s2}{FF\PYGZus{}10}\PYG{l+s+s2}{\PYGZdq{}}\PYG{p}{]}
\PYG{n}{start\PYGZus{}date\PYGZus{}} \PYG{o}{=} \PYG{l+m+mi}{199401190000}
\PYG{n}{end\PYGZus{}date\PYGZus{}}   \PYG{o}{=} \PYG{l+m+mi}{199402191020}
\PYG{n}{x\PYGZus{}coordinate\PYGZus{}} \PYG{o}{=} \PYG{l+m+mf}{6.0941} \PYG{c+c1}{\PYGZsh{} 7 for compare == False}
\PYG{n}{y\PYGZus{}coordinate\PYGZus{}} \PYG{o}{=} \PYG{l+m+mf}{50.7827} \PYG{c+c1}{\PYGZsh{} 51 for compare == False}
\PYG{n}{z\PYGZus{}coordinate\PYGZus{}} \PYG{o}{=} \PYG{l+m+mi}{0} \PYG{c+c1}{\PYGZsh{} not needed for now (maybe in future)}
\PYG{n}{k\PYGZus{}factor\PYGZus{}} \PYG{o}{=} \PYG{l+m+mi}{7} \PYG{c+c1}{\PYGZsh{} how many station are you looking for around your location? 7 means, it will find 7 next stations for your location}
\PYG{n}{compare\PYGZus{}station\PYGZus{}} \PYG{o}{=} \PYG{l+s+s2}{\PYGZdq{}}\PYG{l+s+s2}{wind\PYGZus{}00003}\PYG{l+s+s2}{\PYGZdq{}} \PYG{c+c1}{\PYGZsh{} needed for comparing (don\PYGZsq{}t forget to set the prefix (wind\PYGZus{})}
\PYG{n}{type\PYGZus{}of\PYGZus{}data\PYGZus{}} \PYG{o}{=} \PYG{l+s+s2}{\PYGZdq{}}\PYG{l+s+s2}{wind}\PYG{l+s+s2}{\PYGZdq{}}
\PYG{n}{type\PYGZus{}of\PYGZus{}time\PYGZus{}} \PYG{o}{=} \PYG{l+s+s2}{\PYGZdq{}}\PYG{l+s+s2}{historical}\PYG{l+s+s2}{\PYGZdq{}}
\PYG{n}{dwd} \PYG{o}{=} \PYG{n}{main\PYGZus{}dwd}\PYG{p}{(}\PYG{n}{local\PYGZus{}domain}\PYG{o}{=}\PYG{n}{local\PYGZus{}domain\PYGZus{}}\PYG{p}{,}
               \PYG{n}{type\PYGZus{}of\PYGZus{}data}\PYG{o}{=}\PYG{n}{type\PYGZus{}of\PYGZus{}data\PYGZus{}}\PYG{p}{,}
               \PYG{n}{type\PYGZus{}of\PYGZus{}time}\PYG{o}{=}\PYG{n}{type\PYGZus{}of\PYGZus{}time\PYGZus{}}\PYG{p}{,}
               \PYG{n}{start\PYGZus{}date}\PYG{o}{=}\PYG{n}{start\PYGZus{}date\PYGZus{}}\PYG{p}{,}
               \PYG{n}{end\PYGZus{}date}\PYG{o}{=}\PYG{n}{end\PYGZus{}date\PYGZus{}}\PYG{p}{,}
               \PYG{n}{compare\PYGZus{}station}\PYG{o}{=}\PYG{n}{compare\PYGZus{}station\PYGZus{}}\PYG{p}{,}
               \PYG{n}{x\PYGZus{}coordinate}\PYG{o}{=}\PYG{n}{x\PYGZus{}coordinate\PYGZus{}}\PYG{p}{,}
               \PYG{n}{y\PYGZus{}coordinate}\PYG{o}{=}\PYG{n}{y\PYGZus{}coordinate\PYGZus{}}\PYG{p}{,}
               \PYG{n}{z\PYGZus{}coordinate}\PYG{o}{=}\PYG{n}{z\PYGZus{}coordinate\PYGZus{}}\PYG{p}{,}
               \PYG{n}{k\PYGZus{}factor}\PYG{o}{=}\PYG{n}{k\PYGZus{}factor\PYGZus{}}\PYG{p}{,}
               \PYG{n}{looking\PYGZus{}for}\PYG{o}{=}\PYG{n}{looking\PYGZus{}for\PYGZus{}}\PYG{p}{)}

\PYG{n}{dwd}\PYG{o}{.}\PYG{n}{main\PYGZus{}plotter\PYGZus{}data}\PYG{p}{(}\PYG{n}{qn\PYGZus{}weight}\PYG{o}{=}\PYG{k+kc}{False}\PYG{p}{,} \PYG{n}{distance\PYGZus{}weight}\PYG{o}{=}\PYG{k+kc}{True}\PYG{p}{,} \PYG{n}{compare}\PYG{o}{=}\PYG{k+kc}{False}\PYG{p}{,} \PYG{n}{no\PYGZus{}plot}\PYG{o}{=}\PYG{k+kc}{False}\PYG{p}{)}
\PYG{n}{dwd}\PYG{o}{.}\PYG{n}{main\PYGZus{}plotter\PYGZus{}data}\PYG{p}{(}\PYG{n}{qn\PYGZus{}weight}\PYG{o}{=}\PYG{k+kc}{False}\PYG{p}{,} \PYG{n}{distance\PYGZus{}weight}\PYG{o}{=}\PYG{k+kc}{True}\PYG{p}{,} \PYG{n}{compare}\PYG{o}{=}\PYG{k+kc}{False}\PYG{p}{,} \PYG{n}{no\PYGZus{}plot}\PYG{o}{=}\PYG{k+kc}{True}\PYG{p}{)}
\PYG{n}{dwd}\PYG{o}{.}\PYG{n}{main\PYGZus{}plotter\PYGZus{}data}\PYG{p}{(}\PYG{n}{qn\PYGZus{}weight}\PYG{o}{=}\PYG{k+kc}{False}\PYG{p}{,} \PYG{n}{distance\PYGZus{}weight}\PYG{o}{=}\PYG{k+kc}{True}\PYG{p}{,} \PYG{n}{compare}\PYG{o}{=}\PYG{k+kc}{True}\PYG{p}{,} \PYG{n}{no\PYGZus{}plot}\PYG{o}{=}\PYG{k+kc}{True}\PYG{p}{)}
\PYG{c+c1}{\PYGZsh{}Making calculations or/and plots for your data. (available methods: qn\PYGZus{}weight, distance\PYGZus{}weight, average)}

\PYG{c+c1}{\PYGZsh{} qn\PYGZus{}weight: will use the quality of data (qn) as weight.}
\PYG{c+c1}{\PYGZsh{} distance\PYGZus{}weight: will use the distance as weight.}
\PYG{c+c1}{\PYGZsh{} compare: will compare your calculation with the station you choosed.}
\PYG{c+c1}{\PYGZsh{} If qn\PYGZus{}weight == False, distance\PYGZus{}weight == False, it will use standard\PYGZhy{}average calc. method.}
\PYG{c+c1}{\PYGZsh{} If you want just to see the numbers of your calculation, you can set no\PYGZus{}plot == True (will be faster).}
\end{sphinxVerbatim}

\begin{sphinxadmonition}{note}{Note:}
\sphinxAtStartPar
If you are comparing your calculations, make sure, that the x\_coordinate and y\_coordinate are exactly the same as for the station you are comparing with.
\end{sphinxadmonition}

\noindent\sphinxincludegraphics[width=1000\sphinxpxdimen]{{wind_FF_10_no_compare}.png}


\chapter{Indices and tables}
\label{\detokenize{quick_start:indices-and-tables}}\begin{itemize}
\item {} 
\sphinxAtStartPar
\DUrole{xref,std,std-ref}{genindex}

\item {} 
\sphinxAtStartPar
\DUrole{xref,std,std-ref}{modindex}

\item {} 
\sphinxAtStartPar
\DUrole{xref,std,std-ref}{search}

\end{itemize}


\chapter{Mathematical equations}
\label{\detokenize{math_equations:mathematical-equations}}\label{\detokenize{math_equations::doc}}

\section{Calculations}
\label{\detokenize{math_equations:calculations}}
\begin{sphinxVerbatim}[commandchars=\\\{\}]
\PYG{k+kn}{import} \PYG{n+nn}{os}
\PYG{k+kn}{from} \PYG{n+nn}{DwdMain} \PYG{k+kn}{import} \PYG{n}{main\PYGZus{}dwd}
\PYG{n}{local\PYGZus{}domain\PYGZus{}} \PYG{o}{=} \PYG{l+s+sa}{r}\PYG{l+s+s2}{\PYGZdq{}}\PYG{l+s+s2}{YOUR\PYGZus{}PATH/}\PYG{l+s+s2}{\PYGZdq{}}
\PYG{n}{os}\PYG{o}{.}\PYG{n}{chdir}\PYG{p}{(}\PYG{n}{local\PYGZus{}domain\PYGZus{}}\PYG{p}{)}
\PYG{n}{looking\PYGZus{}for\PYGZus{}} \PYG{o}{=} \PYG{p}{[}\PYG{l+s+s2}{\PYGZdq{}}\PYG{l+s+s2}{FF\PYGZus{}10}\PYG{l+s+s2}{\PYGZdq{}}\PYG{p}{]}
\PYG{n}{start\PYGZus{}date\PYGZus{}} \PYG{o}{=} \PYG{l+m+mi}{199401190000}
\PYG{n}{end\PYGZus{}date\PYGZus{}}   \PYG{o}{=} \PYG{l+m+mi}{199402191020}
\PYG{n}{x\PYGZus{}coordinate\PYGZus{}} \PYG{o}{=} \PYG{l+m+mf}{6.0941} \PYG{c+c1}{\PYGZsh{} 7 for compare == False}
\PYG{n}{y\PYGZus{}coordinate\PYGZus{}} \PYG{o}{=} \PYG{l+m+mf}{50.7827} \PYG{c+c1}{\PYGZsh{} 51 for compare == False}
\PYG{n}{z\PYGZus{}coordinate\PYGZus{}} \PYG{o}{=} \PYG{l+m+mi}{0} \PYG{c+c1}{\PYGZsh{} not needed for now (maybe in future)}
\PYG{n}{k\PYGZus{}factor\PYGZus{}} \PYG{o}{=} \PYG{l+m+mi}{7} \PYG{c+c1}{\PYGZsh{} how many station are you looking for around your location? 7 means, it will find 7 next stations for your location}
\PYG{n}{compare\PYGZus{}station\PYGZus{}} \PYG{o}{=} \PYG{l+s+s2}{\PYGZdq{}}\PYG{l+s+s2}{wind\PYGZus{}00003}\PYG{l+s+s2}{\PYGZdq{}} \PYG{c+c1}{\PYGZsh{} needed for comparing (don\PYGZsq{}t forget to set the prefix (wind\PYGZus{})}
\PYG{n}{type\PYGZus{}of\PYGZus{}data\PYGZus{}} \PYG{o}{=} \PYG{l+s+s2}{\PYGZdq{}}\PYG{l+s+s2}{wind}\PYG{l+s+s2}{\PYGZdq{}}
\PYG{n}{type\PYGZus{}of\PYGZus{}time\PYGZus{}} \PYG{o}{=} \PYG{l+s+s2}{\PYGZdq{}}\PYG{l+s+s2}{historical}\PYG{l+s+s2}{\PYGZdq{}}
\PYG{n}{dwd} \PYG{o}{=} \PYG{n}{main\PYGZus{}dwd}\PYG{p}{(}\PYG{n}{local\PYGZus{}domain}\PYG{o}{=}\PYG{n}{local\PYGZus{}domain\PYGZus{}}\PYG{p}{,}
               \PYG{n}{type\PYGZus{}of\PYGZus{}data}\PYG{o}{=}\PYG{n}{type\PYGZus{}of\PYGZus{}data\PYGZus{}}\PYG{p}{,}
               \PYG{n}{type\PYGZus{}of\PYGZus{}time}\PYG{o}{=}\PYG{n}{type\PYGZus{}of\PYGZus{}time\PYGZus{}}\PYG{p}{,}
               \PYG{n}{start\PYGZus{}date}\PYG{o}{=}\PYG{n}{start\PYGZus{}date\PYGZus{}}\PYG{p}{,}
               \PYG{n}{end\PYGZus{}date}\PYG{o}{=}\PYG{n}{end\PYGZus{}date\PYGZus{}}\PYG{p}{,}
               \PYG{n}{compare\PYGZus{}station}\PYG{o}{=}\PYG{n}{compare\PYGZus{}station\PYGZus{}}\PYG{p}{,}
               \PYG{n}{x\PYGZus{}coordinate}\PYG{o}{=}\PYG{n}{x\PYGZus{}coordinate\PYGZus{}}\PYG{p}{,}
               \PYG{n}{y\PYGZus{}coordinate}\PYG{o}{=}\PYG{n}{y\PYGZus{}coordinate\PYGZus{}}\PYG{p}{,}
               \PYG{n}{z\PYGZus{}coordinate}\PYG{o}{=}\PYG{n}{z\PYGZus{}coordinate\PYGZus{}}\PYG{p}{,}
               \PYG{n}{k\PYGZus{}factor}\PYG{o}{=}\PYG{n}{k\PYGZus{}factor\PYGZus{}}\PYG{p}{,}
               \PYG{n}{looking\PYGZus{}for}\PYG{o}{=}\PYG{n}{looking\PYGZus{}for\PYGZus{}}\PYG{p}{)}
\end{sphinxVerbatim}


\subsection{method: standard average}
\label{\detokenize{math_equations:method-standard-average}}
\begin{sphinxVerbatim}[commandchars=\\\{\}]
\PYG{n}{dwd}\PYG{o}{.}\PYG{n}{main\PYGZus{}plotter\PYGZus{}data}\PYG{p}{(}\PYG{n}{qn\PYGZus{}weight}\PYG{o}{=}\PYG{k+kc}{False}\PYG{p}{,} \PYG{n}{distance\PYGZus{}weight}\PYG{o}{=}\PYG{k+kc}{False}\PYG{p}{,} \PYG{n}{compare}\PYG{o}{=}\PYG{k+kc}{True}\PYG{p}{,} \PYG{n}{no\PYGZus{}plot}\PYG{o}{=}\PYG{k+kc}{False}\PYG{p}{)}
\end{sphinxVerbatim}

\sphinxAtStartPar
\sphinxstylestrong{if compare == False}
\begin{equation*}
\begin{split}\begin{eqnarray}
     A =
     \left[ \begin{array}{rrr}
     x_{11} & x_{12} & ... & x_{1j}\\
     x_{21} & x_{22} & ... & x_{2j}\\
     ...    & ... & ...    & ...   \\
     x_{i1} & x_{i2} & ... & x_{ij} \\
     \end{array}\right]\\
     y(i) = \frac{\displaystyle\sum\limits_{j=1}^{kfactor} x_{ij}}{kfactor}\\
     \triangle time = endDate - startDate\\
     A:= Matrix \ with \ your \ data \\
     kfactor := number \ of \ stations \ around \ your \ location\\
     y(i) := function \ for \ every \ point \ in \ your \  \triangle time \\
 \end{eqnarray}\end{split}
\end{equation*}
\sphinxAtStartPar
\sphinxstylestrong{if compare == True}
\begin{align*}\!\begin{aligned}
A =
  \left[ \begin{array}{rrr}
  x_{11} & x_{12} & ... & x_{1j}\\
  x_{21} & x_{22} & ... & x_{2j}\\
  ...    & ... & ...    & ...   \\
  x_{i1} & x_{i2} & ... & x_{ij} \\
  \end{array}\right]\\
  y(i) = \frac{\displaystyle\sum\limits_{j=2}^{kfactor-1} x_{ij}}{kfactor-1}\\\\
y = \left(\begin{array}{c}y(1)\\ y(2)\\ y(3) \\ ...\\ y(n)\end{array}\right)\\\\
c = \left(\begin{array}{c}x_{11}\\ x_{21}\\ x_{31} \\ ...\\ x_{n1}\end{array}\right)\\\\
\Rightarrow diff = \left(\begin{array}{c}
 \arrowvert y(1) - x_{11} \arrowvert\\
 \arrowvert y(2) - x_{21} \arrowvert\\
 \arrowvert y(3) - x_{31} \arrowvert\\
 ...\\
 \arrowvert y(n) - x_{n1} \arrowvert\\
 \end{array}\right) = \left(\begin{array}{c}
 d_{1}\\
 d_{2}\\
 d_{3} \\
 ...\\
 d_{n}\end{array}\right)\\ \\\\
\Rightarrow avgdiff = \frac{\displaystyle\sum\limits_{i=1}^{n} d_{i}}{n}\\\\
\triangle time = endDate - startDate\\
  A:= Matrix \ with \ your \ data \\
  kfactor := number \ of \ stations \ around \ your \ location\\
  y(i) := averageFunction \ for \ every \ point \ in \ your \  \triangle time \\
  y := averageVector \\
  c := data \ of \ the \ station \ you \ are \ comparing \ with\\\\
\end{aligned}\end{align*}
\begin{sphinxVerbatim}[commandchars=\\\{\}]
\PYG{n}{dwd}\PYG{o}{.}\PYG{n}{main\PYGZus{}plotter\PYGZus{}data}\PYG{p}{(}\PYG{n}{qn\PYGZus{}weight}\PYG{o}{=}\PYG{k+kc}{False}\PYG{p}{,} \PYG{n}{distance\PYGZus{}weight}\PYG{o}{=}\PYG{k+kc}{True}\PYG{p}{,} \PYG{n}{compare}\PYG{o}{=}\PYG{k+kc}{True}\PYG{p}{,} \PYG{n}{no\PYGZus{}plot}\PYG{o}{=}\PYG{k+kc}{False}\PYG{p}{)}
\end{sphinxVerbatim}

\sphinxAtStartPar
If \(\sigma_{1}\) equals \(\sigma_{2}\) then etc, etc.

\begin{sphinxVerbatim}[commandchars=\\\{\}]
\PYG{n}{dwd}\PYG{o}{.}\PYG{n}{main\PYGZus{}plotter\PYGZus{}data}\PYG{p}{(}\PYG{n}{qn\PYGZus{}weight}\PYG{o}{=}\PYG{k+kc}{False}\PYG{p}{,} \PYG{n}{distance\PYGZus{}weight}\PYG{o}{=}\PYG{k+kc}{False}\PYG{p}{,} \PYG{n}{compare}\PYG{o}{=}\PYG{k+kc}{True}\PYG{p}{,} \PYG{n}{no\PYGZus{}plot}\PYG{o}{=}\PYG{k+kc}{False}\PYG{p}{)}
\end{sphinxVerbatim}

\sphinxAtStartPar
If \(\sigma_{1}\) equals \(\sigma_{2}\) then etc, etc.


\chapter{Indices and tables}
\label{\detokenize{index:indices-and-tables}}\begin{itemize}
\item {} 
\sphinxAtStartPar
\DUrole{xref,std,std-ref}{genindex}

\item {} 
\sphinxAtStartPar
\DUrole{xref,std,std-ref}{modindex}

\item {} 
\sphinxAtStartPar
\DUrole{xref,std,std-ref}{search}

\end{itemize}


\renewcommand{\indexname}{Python Module Index}
\begin{sphinxtheindex}
\let\bigletter\sphinxstyleindexlettergroup
\bigletter{d}
\item\relax\sphinxstyleindexentry{DwdDataPrep}\sphinxstyleindexpageref{DwdDataPrep:\detokenize{module-DwdDataPrep}}
\item\relax\sphinxstyleindexentry{DwdDataScrapper}\sphinxstyleindexpageref{DwdDataScrapper:\detokenize{module-DwdDataScrapper}}
\item\relax\sphinxstyleindexentry{DwdDict}\sphinxstyleindexpageref{DwdDict:\detokenize{module-DwdDict}}
\item\relax\sphinxstyleindexentry{DwdMain}\sphinxstyleindexpageref{DwdMain:\detokenize{module-DwdMain}}
\item\relax\sphinxstyleindexentry{DwdNearNeighbor}\sphinxstyleindexpageref{DwdNearNeighbor:\detokenize{module-DwdNearNeighbor}}
\item\relax\sphinxstyleindexentry{DwdPlotter}\sphinxstyleindexpageref{DwdPlotter:\detokenize{module-DwdPlotter}}
\indexspace
\bigletter{m}
\item\relax\sphinxstyleindexentry{main}\sphinxstyleindexpageref{main:\detokenize{module-main}}
\end{sphinxtheindex}

\renewcommand{\indexname}{Index}
\printindex
\end{document}